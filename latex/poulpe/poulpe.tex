\documentclass[12pt, a4paper]{book}

\usepackage[utf8]{inputenc}
\usepackage[T1]{fontenc}
\usepackage[frenchb]{babel} % Pour changer le pack de police

\usepackage{makeidx}

\usepackage{setspace}
\usepackage[left=2cm,right=2cm,top=2cm,bottom=2cm]{geometry}

\usepackage{fancyhdr}

\fancyhf{}
\pagestyle{fancy}
\lhead{\leftmark}
\chead{}
\lfoot{}
\cfoot{}
\rfoot{\thepage}

\makeindex
\begin{document}

\begin{spacing}{1.5}

\frontmatter

\chapter*{Remerciements}
\thispagestyle{empty}
Avant de présenter ce mémoire, je tiens vraiment à remercier les deux sujets, envers qui j'adresse tout mon respect. Livrer son histoire personnelle n'est pas quelque chose d’évident et je les remercie encore pour la confiance qu’ils ont su me donner

Je souhaite également remercier la psychologue de l'institution, Mme Hermet, pour son accueil et sa bienveillance.

Je remercie spécialement Mme Harrati qui a pu me donner les indications et les conseils nécessaires, pour aller toujours plus loin dans la réalisation de ce mémoire.

Je remercie également Mr Vavassori pour avoir accepté d’être mon assesseur.

Une pensée particulière pour Gautier qui a su m'encourager tout au long de l'élaboration de ce mémoire, et qui a su m’insuffler le courage nécessaire, par sa patience et sa confiance en moi.

Enfin, je remercie ma famille, Nicolas, Maïwenn et Lauriane qui n'ont jamais cessé de me soutenir.

\chapter*{Résumé}
\thispagestyle{empty}

Notre recherche s’articule autour de la question suivante : La fonction de pare-excitation de l’institution se révèle globalement efficiente, comment peut-on comprendre alors sa mise en échec ponctuelle par le sujet adolescent violent ?

Notre hypothèse générale suppose que la mise en échec de la fonction de pare-excitation de l’institution peut être interprétée comme la résultante d’une tentative de symbolisation de la séparation et de l’individuation chez l’adolescent dans le passage à l’acte violent. Cette tentative de symbolisation est permise par l’institution qui se constitue en tant qu’objet-mère. Elle réactualise dans ce cas la relation primaire objectale qui n’a pas pu permettre la mise en place de la fonction symbolique. Afin de répondre à cette question et de vérifier notre hypothèse générale, nous avons alors rencontré deux sujets à qui nous avons fait passer un entretien et un test projectif. Suite à l’analyse clinique des données, nos résultats démontrent une difficulté d’élaboration et une dépendance à l’institution qui se constitue en tant que figuration psychique. L’échec de la fonction de pare-excitation, résulterait d’une réponse à un sentiment d’emprise de l’institution, chez le sujet, qui ne pourra élaborer symboliquement que par le passage à l’acte.

\vspace{1cm}

\underline{Mots clefs :} Pare-excitation, Fonction symbolique, Passage à l’acte, Institution, Adolescent, Séparation et individuation, Objet-mère, Relation primaire objectale


\section*{Abstract}
Our general hypothesis assumes that the failure of the 'excitation spare function' can be regarded as the result of an attempt of symbolization of the separation and individuation of the teenager when taking violent action. This attempt of symbolization is enabled by the institution which establishes itself as 'mother-object'. In this case, it updates the primal object relationship that could not enable the setting up of the symbolical function. In order to answer this question and to test our general hypothesis, we have met two persons that we have subjected to an interview and a projective test. Following the data clinical analysis, our results show a difficulty in development and a dependance on the institution which constitutes itself as psychic representation.
The failure of the 'excitation spare function' would then be the result of an answer to the feeling of grip, driven by the institution, of the subject, who will only be able to develop symbolically by commitment to action.

\vspace{1cm}

\thispagestyle{empty}
\underline{Keywords :} Excitation spare function, symbolical function, taking violent action, institution, teenager, separation and individuation, mother-objet, primal object relationship
\chapter*{Introduction}

\thispagestyle{empty}
L'institution, par ses règles et son cadre institutionnel incarnant une instance symbolique contenante, peut parvenir à aider les personnes qui constituent son  public à contenir leurs débordements pulsionnels. Cependant, certaines d'entre elles semblent n'y pas parvenir à certains moments.

L'objectif de ce mémoire est de comprendre les enjeux psychiques qui s'instaurent entre les processus psychiques de l'institution et ceux de l'usager afin de mieux comprendre cet échec de la fonction de pare-excitation de  l'institution. Situer le passage à l'acte dans la dimension psychique du sujet, le relier à la dimension symbolique de l'Institution permettra de comprendre les épisodes de mise en échec de celle-ci  à se constituer en pare-excitation.

Dans notre mémoire, nous choisirons une population d’adolescents souffrants de troubles du comportement accueillis par un Institut Educatif et Pédagogique (ITEP). L'adolescence se définit par des bouleversements psychiques et pulsionnels et par une réactualisation des angoisses archaïques et  de leur mode relationnel primaire. Cette période donne à voir non seulement un débordement pulsionnel propre à la phase de l'adolescence mais un envahissement de problématiques psychiques ancrées dans leurs trajectoires de vie. L'étude de l'adolescent inscrit dans le passage à l'acte nous permettra de mettre en relief ses défaillances de mentalisation et de symbolisation qui risqueront de le mettre dans l'incapacité à contenir ses débordements pulsionnels.  Or, l'intégration des fonctions symboliques se réalise durant la prime enfance. Le passage à l'acte, malgré l'illisibilité immédiate de son sens, peut constituer un moyen pour le sujet d'accéder à la fonction symbolique.

Selon nos apports théoriques, l'institution tendrait à compenser ce déficit symbolique. Malgré cette ressource considérée comme contenante, pourquoi le sujet se livre t-il malgré tout  au passage à l'acte ? Le constat des difficultés institutionnelles à contenir l'excitation de certains usagers témoigne t-il d’une défaillance de cette instance représentante de la loi et des moyens mis en place pour prévenir le passage à l'acte ? Quels sont donc pour le sujet la fonction et le sens de ses passages à l’cte dans un contexte institutionnel qui se constitue pourtant en  pare-excitation auprès de la plupart des usagers ?   

Afin de recueillir des éléments de réponse à ces interrogations,  nous nous réfèrerons à un champ pluridisciplinaire large tel que la psychopathologie, la psychologie clinique et la psychanalyse.

Dans une première partie théorique, nous évoquerons la clinique de ladolescence, ce qui permettra de rendre compte des réaménagements psychiques propres à cette période et de l'importance de la réactualisation de la dimension séparation/individuation qui s'origine dans la première enfance. Une clinique de l'acte permettra de mettre en évidence le déficit de symbolisation et d'élaboration chez les sujets inscrits dans le passage à l'acte. Celui-ci pouvant être compris comme une tentative de compensation d’une fragilité narcissique. Et pour clore cette partie, une clinique de l'institution permettra d'étayer notre théorie sur la fonction de pare-excitation, d'éclairer en quoi l'institution pourrait se constituer dans cette fonction. 

Notre investigation va ainsi articuler le défaut de symbolisation de cette population et la fonction pare-excitatrice de l'institution. Dans le cadre de la partie méthodologique, nous exposerons notre problématique, notre hypothèse ainsi que notre protocole de recherche. Une passation du test de Rorschach et d’un entretien semi-directif permettront de répondre aux trois aspects de notre problématique : leur parcours de vie d’adolescent, leur rapport à l'acte et enfin à l'institution. Ces résultats obtenus permettront d'éclairer la  dynamique psychique de ces deux adolescents en prenant en compte leurs relations objectales et narcissiques, le traitement de leurs motions pulsionnelles et enfin les effets de leurs figurations psychiques de l'Institution. La dernière partie de ce mémoire sera consacrée à la discussion de la validité de nos résultats et nos conclusions sur la compréhension de la mise en échec de la fonction de pare-excitation de l'Institution par le sujet adolescent dans le passage à l'acte.

Cette investigation nous amènera à la question centrale de ce mémoire : \textbf{la fonction de pare-excitation de l’institution se révèle globalement efficiente, alors comment peut-on comprendre sa mise en échec ponctuelle par le sujet adolescent qui s'inscrit dans le passage à l'acte violent ?}

 \addtocontents{toc}{\protect\thispagestyle{empty}}
\tableofcontents
 \addtocontents{toc}{\protect\thispagestyle{empty}}
\thispagestyle{empty}

\mainmatter

\part{Partie théorique}

\chapter{Clinique de l'adolescent}

L'adolescence est une période transitoire où se concrétise à la fois un bouleversement corporel, mais également un revirement psychique qui impliquera d'importantes transformations. Celles-ci auront un impact majeur sur l'état psychique de l'individu : son Moi en ressortira fragilisé. Selon la psychanalyse, l'adolescent va exploiter des moyens de défense afin d'essayer de préserver ce Moi déstabilisé. En premier lieu de cette partie consacrée à l'adolescence, il s'agira d'identifier les causes de cette déstabilisation psychique selon le référentiel théorique de la psychopathologie clinique psychanalytique de l'adolescence, puis de rendre compte des aménagements défensifs opérés par l'adolescent. Pour terminer cette partie, nous évoquerons que le mode de relation objectale du sujet adolescent, étroitement lié au développement affectif de celui-ci, impactera son fonctionnement psychique, qui mettra en œuvre un type de défense spécifique.

\section{Les remaniements psychiques de l'adolescence : les principaux aspects dynamiques}

Le passage de l'enfance à l'âge adulte va être marqué par la découverte de l'élaboration psychique. Cependant, ce processus de subjectivation est régit par des conflits psychiques qui vont apparaître lors des phases dynamiques propres à tout individu adolescent (Marcelli\index{Marcelli} et Braconnier\index{Braconnier}, 1992). Ces phases correspondent aux modifications pulsionnelles, corporelles qui vont amener à une conception nouvelle du rapport au corps, à un remaniement narcissique et identitaire tel l'instauration d'un idéal du Moi mais également par un investissement accru de supports identificatoires.

\subsection{L'excitation sexuelle}

Du point de vue biologique, la période de l'adolescence est signalée dans un premier temps par une modification corporelle importante. Ces changements seront provoqués par une poussée hormonale, phénomène déterminant de la puberté.

Du point de vue psychanalytique, cette poussée hormonale accroît l'activité  libidinale. Celle-ci, définie par un aspect économique et dynamique, va venir fragiliser le Moi (dans son rôle de pare-excitation) au moment où apparaît le changement du but sexuel. Freud\index{Freud} (1905),  explique que pendant l'enfance la pulsion sexuelle est auto-érotique puis, à l'adolescence,  cette pulsion trouvera l'objet sexuel et un nouveau but : celui de la décharge des pulsions sexuelles au service de la reproduction. C'est ce changement de but et d'objet sexuel qui va entraîner une explosion libidinale, une éruption pulsionnelle génitale et un mouvement de régression vers les pulsions prégénitales. S'ajoutant à ces changements économiques, des changements d'ordre dynamique ont lieu. Le conflit intérieur de l'adolescent n'est pas seulement la réplique du conflit Œdipien car sont pointés aussi des conflits entre le Moi Idéal réactualisé et le Moi déstabilisé. D'ailleurs, Freud\index{Freud} (1905) définit la puberté comme une « déchirure de l'enveloppe infantile. » Sa fille Anna (1936) complète sa théorie en précisant que l'aboutissement de la puberté ne repose pas sur la victoire de la puissance des pulsions mais plutôt sur la tolérance du Moi à l'égard de ces pulsions. Ce n'est donc pas la consistance du ça qui se modifie mais plutôt les relations qu'établit le Moi avec le ça.
La transformation corporelle sera permise, du point de vue biologique, par une recrudescence des phénomènes hormonaux qui vont, comme nous venons de l'exposer, générer un flux d'excitation sexuelle. L'adolescent va devoir y répondre en aménageant un mode de réponse défensif dans le but de protéger son Moi affaibli par cette puissance pulsionnelle. Cette transformation corporelle pourra engendrer violence et traumatisme à l'enfant qui accède à la puberté. Pirlot\index{Pirlot} (2001),  affirme que cette violence et ce traumatisme sont marqués par une difficulté à penser cette transformation. Pour Laplanche\index{Laplanche} et Pontalis\index{Pontalis} (1967),  le traumatisme se définit par un flux d'excitation qui est excessif par rapport à la tolérance du sujet et à sa capacité de maîtriser et d'élaborer psychiquement son excitation. En fait, le trauma annihile l'ordre dynamique et topique du psychisme. Selon Pirlot\index{Pirlot} : « Le corps et ses transformations empêchent véritablement de penser. »
L'adolescent essayera donc de mettre en place un moyen de penser autrement cette transformation corporelle, qui aura un impact sur son rapport au corps.

\subsection{La problématique du corps}

La puberté se manifeste par de profondes modifications physiologiques qui ont d'importantes répercussions psychologiques sur la réalité concrète mais également sur le plan imaginaire et symbolique. Gutton\index{Gutton} (1991) marque la différence entre la puberté et le pubertaire. Le pubertaire est à la psyché ce que la puberté est au corps. Ce processus montre la pression biologique sur les trois instances psychiques (Moi, ça, Surmoi). La proximité pour le Moi avec l'énergie du ça (réservoir de la libido) explique alors la violence aussi bien des conduites que des comportements défensifs de l'adolescent. Toutes ces modifications physiques, biologiques et physiologiques ont un impact fondamental sur le processus de l'adolescence. Elles permettront l'accession à la sexualité génitale. 

Pour Freud\index{Freud} (1905) « avec le commencement de la puberté, apparaissent des transformations qui amèneront la vie sexuelle infantile à sa forme définitive et normale ». La pulsion va découvrir l'objet sexuel chez l'autre, les diverses zones érogènes dites partielles vont se subordonner au primat de la zone génitale. La jouissance sexuelle permettra d'accéder au « plaisir terminal »  opposé aux plaisirs préliminaire. En fait, le processus de puberté entraîne de profondes modifications corporelles mais également pulsionnelles, ce qui résultera d'une conception nouvelle du rapport au corps. Malgré la pression que ces bouleversements vont exercer sur les trois instances psychiques, ce qui aura pour conséquence une fragilité, le corps sera investi en tant que  représentant symbolique de soi ainsi que porteur d'une  expression sociale de son sentiment d'identité. L'adolescent mettra en place un réaménagement narcissique pour protéger cette fragilité.

\subsection{Le Narcissisme}

Suite à ces mutations corporelles abruptes qui provoqueront un nouveau rapport à son propre corps, l'adolescent doit choisir des objets nouveaux mais doit aussi se choisir soi-même en tant qu'objet d'intérêt, de respect et d'estime. La modification du narcissisme à l'adolescence se manifeste par une augmentation quantitative, mais elle est également répartie différemment sur le plan dynamique. Kernberg\index{Kernberg} (1975) a situé les conduites dites « narcissiques » dans une continuité allant des conduites les plus « normales » aux plus « pathologiques ». Le premier mode se traduit par une augmentation de l'investissement libidinal de soi et sa coexistence avec un investissement libidinal persistant des objets. Le deuxième mode, se caractérise par une identification pathologique de soi aux objets infantiles. Il n'y a plus là de mélange d'investissements narcissiques et d'investissements objectaux. Le dernier mode, le plus pathologique, se manifeste par la conservation d'un soi grandiose avec projection d'un soi primitif omnipotent sur l'objet. C'est dans la direction de ce dernier mode que Marcelli\index{Marcelli} et Braconnier\index{Braconnier} (1992), définissent le narcissisme dit « pathologique » par deux conduites : un désintérêt à l'égard du monde extérieur (l'égoïsme) et par une image de soi grandiose (mégalomanie). Kestemberg\index{Kestemberg} (1972) conserve l'idée de cette image de soi grandiose qui va servir de soutien narcissique comme « Idéal du Moi », image satisfaisante de soi qui sera mis en place par l'adolescent, dans le but de préserver son Moi psychique.

\subsection{La place de l'Idéal du Moi à l'adolescence}

Laufer (1964) fait la relation entre l'apparition de l'Idéal du Moi et le déclin du conflit œdipien en même temps que celui du Surmoi. Les systèmes d'identifications et d'intériorisations doivent avoir acquis une stabilité suffisante : « avant que se soit faite l'internalisation, les précurseurs de l'idéal du Moi sont encore relativement instables, en partie dépendants des sources extérieures. » Laufer poursuit la définition de l'Idéal du Moi comme « la partie du Surmoi qui contient les images et les attributs que le Moi s'efforce d'acquérir afin de rétablir l'équilibre narcissique. » L'adolescent va donc se servir de l'extérieur, du groupe des pairs comme support identificatoire afin de se gratifier narcissiquement, et par la même de renforcer son Moi appauvri. Laufer souligne le conflit intrapsychique provoqué par le dualisme des attentes entre l'Idéal œdipien et le groupe de pairs.

Freud\index{Freud} (1914) affirme: « l'adolescent se trouve confronté aux nouvelles espérances que le monde extérieur met en lui (et en premier lieu ses congénères) et il s'identifie à elles. Ce sont là des identifications du moi, mais elles sont ressenties comme étant du même ordre que les premières exigences intériorisées, et c'est dans ce sens que je les considère comme faisant partie de l'Idéal du Moi à l'adolescence. »

L'adolescent, suite à ses modifications corporelles et pulsionnelles, va se réapproprier une nouvelle image de son corps, et tous ces changements brutaux vont amenuiser la stabilité du Moi. Afin de défendre ce Moi, l'adolescent rectifiera son investissement narcissique par la mise en place d'un idéal du Moi, établi par des supports identificatoires extérieurs, qui permettront de développer sa propre identité.

\subsection{Identité et Identification}

Le phénomène de l'adolescence est marqué par les rejets des identifications antérieures que  constituent les objets parentaux, comme le définit Kestemberg\index{Kestemberg} (1981) c'est ce « mouvement de rupture » qui va marquer le processus d'identification. Cependant, ce rejet risque d'entraîner une perte d'image de soi en tant qu'être sexué. Le socle de l'identité de l'adolescent est alors menacé. Cette prise de distance radicale  avec les identifications au parent de même sexe mais surtout au parent de sexe opposé, suscite ainsi une angoisse concernant sa propre identité. C'est pour cela que l'adolescent va faire appel à des expériences qui lui permettront d'investir de nouveaux supports identificatoires. Ces interactions avec le monde extérieur viendront remodeler ces identifications, donc son identité et d'ailleurs comme le précise Kestemberg\index{Kestemberg} (1962), chez l'adolescent : « Identité et Identification sont alors pratiquement un seul et même mouvement ».

Nous constatons que les aspects dynamiques des modifications pulsionnelles et corporelles ont des répercussions considérables sur l'intégrité du Moi de l'adolescent. Dans le but de préserver ce Moi, l'adolescent va mettre en place des réaménagements narcissiques, tels un  idéal du Moi et un investissement de supports identificatoires. Cependant, comme le précise Kestemberg\index{Kestemberg} (1981), ces réaménagements nécessitent un  « mouvement de rupture » par un « rejet des identifications antérieures que constituent les objets parentaux».  En effet, la problématique de la séparation et de l'individuation paraît centrale chez l'adolescent, comme nous allons le voir par suite avec la théorie de Blos\index{Blos} (1967).

\section{Les remaniements psychiques de l'adolescence : La question de la séparation et de l'individuation}

\subsection{De la petite enfance à l'adolescence}

Mahler\index{Mahler} et Blos\index{Blos} (1967) mettent en évidence un processus de séparation et d'individuation qui se déroulerait de la naissance à la mort mais qui trouverait des temps forts d'actualisation dans la première enfance et dans l'adolescence. Mahler\index{Mahler} va relier ce concept avec les premières années de la vie, qu'elle définit comme une « phase symbiotique » où une unité duelle réside entre la mère et le nourrisson car ce dernier ne peut pas assurer seul sa propre homéostasie. La fiabilité de la figure maternelle est importante, car l'enfant doit avoir confiance et trouver réassurance auprès de sa mère pour pouvoir peu à peu s'en séparer. 

L'adolescence et la petite enfance sont des périodes de restructuration et de renégociation de lien à l'objet. Pendant la petite enfance, l'enfant se sépare de la mère réelle grâce à l'internalisation de celle-ci, tandis que l'adolescent se sépare au contraire de ses objets internalisés pour l'amener à rencontrer un nouvel objet réel. Cependant, le déplacement du premier objet vers un autre relève de la complexité en raison du caractère narcissique des liens objectaux. Blos\index{Blos} va resituer ce concept lors de la période de l'adolescence, comme « second processus de séparation-individuation ». Bernateau\index{Bernateau} (2008) affirme que la question de la séparation et de l'individuation demeure centrale à l'adolescence, car elle intervient dans le processus de subjectivation mais également dans le processus de différenciation. D'ailleurs, Gutton\index{Gutton} parle d'une « unité narcissique originaire pubertaire » qui souligne la reviviscence de l'archaïque au moment de l'adolescence, qui, selon Bernateau\index{Bernateau} : « n'est pas une répétition mais une ouverture au remaniement ». 

C'est d'ailleurs ce qu'illustre la citation de Kestemberg\index{Kestemberg} (1984) : « Si tout se prépare dans l'enfance, voire la toute petite enfance, peut être même dans les premiers jours de la vie, tout se joue dans l'adolescence. » C'est également ce qu'écrit Winnicott\index{Winnicott} (1961), avant la période de latence, tout être en « bonne santé » a vécu le complexe d'Œdipe, et chaque adolescent a connu des moyens organisés d'accepter et de tolérer les conflits à ces conditions. Toutes les caractéristiques personnelles issues des expériences de l'enfance, qu'elles soient innées ou acquises, vont conduire vers des fixations à des types de vie instinctuelle prégénitaux, à des résidus de la dépendance infantile, et à des formes associées à des échecs de maturation au stade œdipien ou pré-œdipien qui vont prédéterminer des schèmes à l'adolescence. 

\subsection{De l'importance du premier lien objectal primaire à l'advenue des fonctions symboliques}

Afin de permettre la séparation, donc l'individuation, l'investissement narcissique de l'objet primaire est ainsi nécessaire ; c'est ce qu'illustre la théorie de Winnicott\index{Winnicott} au sujet de « la mère suffisamment bonne. » Freud\index{Freud} (1895) fait référence à un « être humain suffisamment proche » qui, à lui seul, sera capable de faire éprouver au sujet l'expérience vécue de satisfaction, lorsque celui-ci subit l'effet d'une excitation endogène ou exogène.

Dans un premier temps, l'enjeu du processus séparation et individuation, est la possibilité de la construction d'une représentation différenciée de soi. L'infans va être amené à différencier l'extérieur de l'intérieur, il pourra se représenter l'objet seulement si celui-ci vient à manquer. Freud\index{Freud} (1920) explique avec son modèle du for-da, une tentative de symbolisation que l'enfant met en place afin de se représenter l'absence de la mère. Dans un second temps, cette construction d'une représentation différenciée de soi, qu'on peut rapprocher de la relation d'objet de Winnicott\index{Winnicott}, précède la différenciation d'objet. Pour Winnicott\index{Winnicott}, ce passage est très complexe, et nécessite la destruction de l'objet. Cette destruction de l'objet, va être élaboré par le concept d'agressivité primaire.

Pour D.W.Winnicott\index{Winnicott} (1939), il existe une agressivité primaire qui est synonyme d'une activité exprimant le « vrai-self » et une tendance innée à la croissance relevant de l'autonomisation du Moi : « la violence est inhérente au développement du self ».  Les échanges narcissiques mère-enfant se construisent dans le Moi. Il affirme que l'agressivité endosse un rôle positif dans le développement de l'enfant car elle permet la différenciation progressive sujet-objet dans la relation mère-enfant. La haine s'avère nécessaire car elle permet la séparation de la fusion maternelle avec l'enfant et cela lui permet d'acquérir un vrai self. La violence et la haine aident à l'advenue des fonctions symboliques. Les pulsions et fantasmes destructeurs de l'enfant permettent de tester la permanence et la résistance de l' « objet-mère ». L'« objet-mère » prend progressivement valeur d'objet détruit/créé et d'objet progressivement objectalisé, différencié de la mère. Ce processus va pouvoir permettre la séparation et la réparation progressive, par la pensée, de ce qu'il fallait détruire pour exister : ces prémices sont indispensables pour mettre en place la fonction symbolique. L'environnement primaire maternel et familial jouera un rôle important dans la mesure où c'est lui qui permettra ou pas de contenir cette agressivité, sans que l'enfant sache ce que sa « mère environnement » sacrifie pour lui car c'est ce non-savoir qui va stimuler l'enfant du côté de sa créativité et de sa pulsion de savoir, celle-ci étant ce qui va rester de l'alliance entre la pulsion d'emprise, pulsion de destruction et pulsion érotique.

Cependant, si la mère-environnement, l'objet-mère, se révèle être fragile, cet enfant aura l'impression que l'objet de ces pulsions sadiques et destructrices n'aura pas survécu à ses attaques meurtrières et à sa haine. L'agressivité primaire basculera alors dans la destructivité au lieu de permettre la création de la représentation. Si l'expérience de la séparation par la haine censée mettre en place la fonction symbolique se révèle défaillante, le processus de mentalisation et de pensée, par là même les assises narcissiques du sujet, seront impactés. 

\subsection{Les effets pathologiques d'une défaillance du premier lien objectale primaire}

Balier\index{Balier} (1988) décrit des dysfonctionnements dans le processus de « perte d'objet », ce moment de défusion de l'enfant d'avec sa mère où il doit renoncer au fantasme selon lequel il maîtrise l'objet de ses pulsions qui est aussi sa source d'amour. L'angoisse de la perte d'objet peut précipiter le sujet dans des comportements violents, qui visent à dénier la dépendance vis-à-vis de cet objet. Il s'agit d'assujettir désespérément celui-ci pour se protéger de sa perte et de la désorganisation qui s'ensuivrait : « Ce que nous retrouvons chez tous nos patients ce sont les processus de défense contre l'effraction du narcissisme par l'objet. » (Balier\index{Balier}, 1988, p. 59)

C'est ainsi que Winnicott\index{Winnicott} (1956) introduit sa théorie de la tendance antisociale. Il l'explique par une « deprivation » qui fait référence à une perte de quelque chose de positif dans l'expérience de l'enfant qui lui a été retiré  par l'absence de la mère ou par une carence de l'objet. Et ce, au moment critique où chez le nourrisson ou le petit enfant, le Moi est en train de parvenir à l'union des pulsions instinctuelles, libidinales et des pulsions agressives. Cette faille de l'environnement affectif de l'enfant au moment de l'apprentissage de l'union des pulsions peut donc entrainer un comportement à tendance antisociale. Par cette « deprivation», cet auteur explique cette « tendance antisociale » comme une tentative de faire en sorte que le monde lui réédifie le cadre qui lui a été brisé. L'enfant qui manifeste cette tendance à détruire est en fait en situation d'appel dirigé vers la « mère-environnement » en espérant que celle-ci viendra contenir ses angoisses. Il espère ainsi que la mère vienne combler les carences qu'il ressent. L'enfant connaît donc un environnement assez bon avant que la mère ne le lâche. L'enfant réclame la restitution de cet environnement, et si celui-ci ne répond pas aux attentes de l'adolescent, va s'installer la répétition d'acte destructeurs et délictueux. Et c'est justement cette problématique qui va nous intéresser par la suite.

Lorsque la relation primaire objectale est sécurisante, le bébé acquiert un sentiment d'intégrité qui donne à son Moi une enveloppe narcissique et un bien être de base. C'est ce que Anzieu\index{Anzieu} (1985) nomme le pare-excitant qui est une des fonctions de l'appareil psychique permettant de défendre l'effraction pulsionnelle endogène tout en contribuant à satisfaire les apports d'excitations nécessaires. Lorsque cette fonction est perturbée, l'acquisition de la délimitation entre l'intérieur et l'extérieur, entre le dehors et le dedans est perturbée. L'angoisse de la fonction de pare-excitation peut prendre deux formes : l'angoisse paranoïde d'intrusion psychique et la survenue d'une angoisse de perte d'objet qui joue le rôle de pare-excitation auxiliaire. C'est pourquoi l'environnement a un rôle important dans la construction du Moi du bébé et le développement de sa vie psychique. Si il y a une faille du contenant, alors le développement psychique est fragilisé et se traduit par des failles de l'établissement et de la construction du système de pare-excitation. Le système de pare-excitation se révélant défaillant, la dépendance à l'environnement permettra de compenser cette fragilité.

\subsection{La dépendance à l'environnement}

Jeammet\index{Jeammet} (1989) définit la violence chez l'adolescent comme une relation dialectique entre la vulnérabilité du Moi générant un sentiment d'insécurité interne, qui va donc dépendre de façon massive de  la réalité externe, et un besoin de défense du Moi par des conduites d'emprise sur autrui. L'enfant naît avec une importante discordance de maturation entre un système sensitif et sensoriel fonctionnel et un système nerveux moteur très immature. Il va ainsi engranger une somme importante d'informations fournie  par son entourage, ce qui favorisera une intense dépendance à l'égard des adultes. L'enfant va donc développer une dépendance motrice et affective qui va introduire le paradoxe au cœur du développement humain, le besoin de dépendance n'ayant d'égal que celui d'autonomie.  Pour être soi, il faut se nourrir des autres mais aussi se différencier d'eux. L'enfant qui est en insécurité interne se sentira très dépendant des autres, et si le lien avec ces personnages tiers n'est pas suffisant, le bébé puis l'enfant risquera de se sentir pris dans des liens de captation affective et de dépendance exagérée. Entre l'angoisse d'abandon et celle d'être sous la dépendance d'autrui, l'enfant est souvent contraint de chercher un compromis par l'insatisfaction, la plainte ou l'opposition par lesquelles il pousse les adultes à s'occuper de lui tout en échappant à leur pouvoir. Plus l'enfant aura intériorisé une relation de confiance et de sécurité avec l'environnement, plus il sera porteur de sa propre capacité à se sécuriser lui-même et à être au contact de ses ressources personnelles, de plaisir notamment, plus il sera autonome et capable de s'ouvrir aux tiers. Mais si l'enfant est en insécurité, il va devenir plus dépendant de l'environnement et du besoin de s'assurer du contrôle de celui-ci. L'enfant qui se sent trop dépendant de son environnement pour assurer sa sécurité va défensivement chercher à rendre son environnement dépendant de lui, notamment par le caprice, qui deviendra à l'adolescence le comportement d'opposition voire de volonté de contrôle d'autrui. Dans le cas d'une carence affective, pour remplacer l'absence d'objet d'attachement, l'enfant développe une activité de quête de sensations qui ont comme caractéristiques d'être toujours douloureuses et avec une dimension autodestructrice.

Jeammet\index{Jeammet} (1989) va rejoindre la théorie de Winnicott\index{Winnicott} (1956), selon laquelle  l'environnement doit donner une occasion nouvelle à la relation au moi puisque l'enfant a perçu que la carence de l'environnement dans le soutien du Moi est justement à l'origine de sa tendance antisociale. Ces symptômes antisociaux qui constituent des tentatives pour reprendre possession de l'environnement sont des signes d'espoir. L'adolescent a donc besoin d'un environnement spécialisé à visées thérapeutiques, capable de répondre, dans la réalité, à l'espoir exprimé par ses symptômes. Avant de renoncer à ses défenses contre une angoisse intolérable, toujours susceptible d'être réactivée par une nouvelle « deprivation », il faut que l'enfant apprenne à faire confiance à son nouvel environnement, lequel doit être stable et objectif. Ces adolescents vont trouver une manière personnelle de faire face aux angoisses liées aux modifications corporelles et pulsionnelles brutales de la puberté et à celles qui ont accompagnées leur histoire traumatique. Ce sont ces adolescents exprimant leur souffrance par la mise en acte au sein du cadre institutionnel que nous étudierons dans ce mémoire. 

\subsection{L'aménagement défensif de l'adolescent en souffrance : La mise en acte}

Freud\index{Freud} (A., 1939) présente la défense comme une activité du Moi destinée à protéger le sujet contre une trop grande exigence pulsionnelle. Les défenses apparaissent sous forme de défenses psychopathologiques quand il existe un conflit aigu entre les différentes instances de la personnalité psychique (ça, Moi, Idéal du Moi, Surmoi) ou entre certaines de ces instances et la réalité. La mise en acte protège l'adolescent du conflit intériorisé et de la souffrance psychique mais entrave toute possibilité de progression, de telle sorte que sa répétition apparaît souvent comme la seule issue. Chez certains adolescents, le manque de verbalisation et de mentalisation de ce conflit psychique, qui va engendrer un manque de symbolisation, entraîne l'agir. Il n'y aura pas de phases entre les pulsions et l'agir. Par exemple, l'adolescent pourra  être divisé par la peur de dépendance, une angoisse d'emprise, et également par la peur d'être abandonné. Cette division résulte d'un conflit psychique, l'adolescent doit gérer le lien et la distance avec l'objet externe. Lorsque celui-ci se sent privé du contrôle de lui-même par l'emprise d'un objet externe, c'est le passage à l'acte qui va permettre de récupérer ce sentiment de maîtrise. Jeammet\index{Jeammet} (1989) met en parallèle les comportements de violence de certains adolescents et les observations de jeunes enfants carencés. Ces enfants là sont à la recherche de sensations exagérées, par des conduites autodestructrices comme l'arrachage de cheveux ou bien de coups sur leurs propres visages, afin de créer un contact avec eux-mêmes. Les adolescents aux comportements violents, ne pouvant aller à la rencontre d'eux mêmes ou bien des adultes, vont aller à la rencontre de la société de façon destructrice, afin de trouver des réponses sur eux-mêmes. 

\vspace{1cm}
\centerline{*}
\vspace{1cm}

Au terme de cette partie, nous pouvons conclure que l'adolescence est  une période de remaniements qui va fragiliser le Moi de l'adolescent. En effet, la poussée hormonale  va provoquer l'excitation sexuelle et la modification corporelle, ce qui va engendrer un nouveau rapport au corps. Ces changements abrupts vont déstabiliser ce Moi. C'est pourquoi l'adolescent va tenter de sauvegarder son narcissisme en dépendant de son environnement, par la mise en place d'un Idéal du Moi et par le processus d'identification, ce qui lui permettra d'étayer sa nouvelle identité. Nous faisons le constat que cette dépendance de l'environnement témoigne d'une fragilité intrapsychique, qui ne lui permet pas de constituer seul sa propre identité. Nous avons vu que l'environnement permet la renégociation du lien à l'objet, non plus par l'internalisation de celui-ci, comme à la prime enfance, mais par un rejet de ces objets internalisés afin de pouvoir rencontrer un objet réel. Cependant, lorsque certains sujets ont rencontré une défaillance lors de la relation primaire objectale, le processus de séparation et d'individuation va se révéler plus complexe. Or, une relation primaire objectale de bonne qualité est nécessaire car elle permet l'accès à la fonction symbolique. N'ayant pas accès à la fonction symbolique qui lui permettrait de se représenter la séparation donc l'individuation, l'adolescent va trouver des solutions afin de pallier ce manque de symbolisation, par l'acte. C'est ce que nous allons étudier par la suite.

\chapter{Clinique de l'acte déviant : la violence}

Comme nous l'avons vu précédemment, la perte de l'objet positif à l'enfance, au moment où le Moi est en train de parvenir à l'union de ses pulsions libidinales et agressives, aura un impact sur son comportement à l'adolescence. Pour D.W.Winnicott\index{Winnicott}, il existe une agressivité primaire qui est nécessaire, car elle permet de se différencier du statut  sujet-objet (c'est dans le Moi où se construit les échanges narcissiques) et donc d'acquérir un vrai self. C'est pour cela que l'environnement familial est important, car c'est lui qui va contenir cette agressivité. Si il échoue, l'enfant basculera dans la violence car il n'aura pas trouvé de limites à son expression. 

\section{De la violence à l'agressivité}

\subsection{Les origines de la violence}

Freud\index{Freud} (1913) considère la violence comme fondement de l'origine de l'humanité. La société humaine serait fondée sur le meurtre collectif, celui du père du totem. Celui-ci serait le symbole de l'intériorisation des lois qui organiserait nos rapports sociaux. Bergeret\index{Bergeret} (1984) souligne l'existence d'un instinct violent fondamental de survie propre à l'humain. Il va définir cette violence instinctive, courant pulsionnel libidinal, comme instinct d'auto-conservation. Et c'est l'acquisition des interdits fondamentaux qui va permettre de canaliser cette violence et d'ouvrir l'accès aux investissements d'objets extérieurs et aux capacités créatrices. Kammerer\index{Kammerer} (2000) va rejoindre cette théorie, en définissant la violence comme universelle, qui s'empare de nous et qui est réveillée dans toutes les situations où nous ne percevons aucun recours possible. Elle accompagne un comportement qui met en actes un scénario fantasmatique inconscient issu d'expériences traumatiques anciennes. Tous ces auteurs s'accordent à dire que la violence est universelle, mais que celle-ci est contenue lors de la prime enfance. Bergeret\index{Bergeret} (1984) indique que si l'acquisition des interdits fondamentaux n'a pas été établie, cet échec de la structuration oedipienne va conduire au désir de nuire à l'autre. La notion d'agressivité résulterait de la forme pervertie et pathologique du complexe d'Œdipe. 

\subsection{La violence comme résultante d'une désintrication pulsionnelle}

Balier\index{Balier} (1988) définit l'agressivité comme pathologique quand est en cause la déliaison des pulsions de mort avec les pulsions de vie. C'est leur désintrication qui crée une tension permanente ainsi appelée une décharge. C'est le retour à la tension zéro qui est recherché. La décharge de l'agressivité devenue libre, cherche à se déverser dans une sorte de contenant qui empêche la désorganisation psychique dont le sujet est menacé.

Bergeret\index{Bergeret} (1984) se pose la question de l'existence d'une agressivité primaire. Elle semble naître avec l'avidité orale toujours insatisfaite et avec le besoin de satisfaction immédiate. Cette agressivité est déjà très complexe dans sa structure. Elle est à la fois activité d'expulsion (corporelle et représentative), motricité à la fois désordonnée et manifestant déjà un désir d'emprise, clivage et déliaison (ça serait là sont rapport avec l'instinct de mort). Bergeret\index{Bergeret} (1984) replace par la suite l'agressivité dans le jeu de l'intrication et de la désintrication. Dans la première théorie des pulsions de Freud\index{Freud}, la haine était en fait liée aux pulsions d'auto-conservation et à la lutte du Moi pour son affirmation. Dans la deuxième théorie, celle-ci lie l'agressivité à l'instinct de mort et au jeu dialectique de celui-ci avec la libido. La pulsion agressive serait la partie de la pulsion de mort tournée vers l'extérieur et investie dans la musculature. Dans cette opération, elle est mise au service de la libido et constituerait ainsi le sadisme. Les pulsions de vie (Éros lié à la sexualité) et les pulsions de mort (Thanatos, lié à la destruction et à l'agression) sont toujours intriquées au point de modifier le but et la réalisation tantôt de l'une tantôt de l'autre. C'est la faillite du dualisme pulsionnel qui est à l'origine de la violence.  Selon Freud\index{Freud}, dans ce dualisme, les pulsions sexuelles sont opposées aux pulsions d'autodestruction : « L'un s'efforce d'englober en des unités toujours plus vastes tout ce qui est ; l'autre cherche à dissocier les combinaisons et à détruire ce qu'a édifié l'Éros. C'est là le travail de jonction et de désintrincation des composantes instinctuelles. » C'est donc la pulsion de mort qui agit, par l'effet de désintrication. Les pulsions sexuelles s'étayent sur les pulsions d'autoconservation (pulsions du Moi). Par exemple, lors des débuts de la vie, l'enfant qui tête sa mère comporte une « prime de plaisir » sur la satisfaction du besoin. La transformation du plaisir, ira libérer les pulsions sexuelles dans le plaisir auto-érotique en détournant les pulsions d'autoconservation de leur but. Ces pulsions d'autoconservation vont préserver l'intégrité du Moi, du narcissisme primaire. Cependant, l'effet d'une désintrication pulsionnelle va faire agir la pulsion de mort sur la pulsion d'autoconservation, afin de préserver le Moi du sujet.

\subsection{La violence comme réponse à une faille narcissique}

Lorsque Freud\index{Freud} élabore sa conception sur le narcissisme, il précise l'importance de ces pulsions d'autoconservation pour la préservation de l'intégrité du Moi (en plus de la satisfaction des besoins). Distinguant narcissisme primaire et narcissisme secondaire, il déclare que le narcissisme primaire est un état précoce où l'enfant s'investit soi-même de toute sa libido, alors que le narcissisme secondaire revient des autres, des objets investis sur le Moi. La libido narcissique investit donc le Moi avant d'investir les objets. Elle investit même le corps et le Soi avant d'investir le Moi.                     Pour Jeammet\index{Jeammet} (1989), la violence mettrait en œuvre un mouvement de « désobjectalisation », consistant en une destruction du lien à l'objet en vue de protéger le narcissisme du sujet. Bergeret\index{Bergeret} (1984) évoque un objet vécu comme « anti-moi » et la mise en œuvre de la violence à l'égard d'autrui viserait à préserver le narcissisme du sujet. Racamier (1992) souligne le caractère anti-narcissique de l'objet, contre lequel le sujet se défend, parce qu'il est révélateur de l'incomplétude du sujet et de sa dépendance à l'égard de l'objet.                                                                  

Cependant, Pirlot\index{Pirlot} (2001) précise qu'une interaction disqualifiante sera perçue par le nourrisson comme une intrusion traumatique qui ne permettra pas l'établissement qualitatif des narcissismes (primaire puis secondaire) et donc du sentiment d'intégrité corporelle psychique. Devenu adolescent, la violence d'un tel sujet traumatisé peut, par la force musculaire (l'emprise) et par la contention des sensations, chercher à défendre ce narcissisme mal construit. Cette fragilité narcissique va menacer l'intégrité du Moi, qui ne va pas constituer un rempart satisfaisant contre la dépression.

\subsection{La violence comme rempart contre la dépression}

Kammerer\index{Kammerer} (2000) explique le passage à l'acte de l'adolescent violent comme un  rempart contre la dépression. Selon lui, l'imago maternelle est souvent toute-puissante et l'imago paternelle ne constitue pas un rempart satisfaisant contre la dépression. La fragilité narcissique de l'adolescent est donc compensée par la mise en place d'un idéal du moi grandiose, réalisé à partir d'identifications superficielles à des personnages héroïques. Il peut être également alimenté par son groupe de pairs où il partagera les mêmes identités d'emprunt. Mais le sujet va se retrouver confronté à une difficulté d'adaptation car son recours au soi grandiose ne va pas forcément correspondre aux épreuves qu'il va traverser. Cet écart va mettre en évidence une remise en question de son idéal du moi grandiose, révélant ainsi une situation d'échec qui pourra le faire basculer dans la dépression. Il ne trouvera comme solution peut-être rien d'autre que le passage à l'acte. Le passage à l'acte, ici, représente un moyen de défense qui va lutter contre la perturbation de l'établissement des premiers liens. Lorsque le sujet n'a pas pu prendre confiance dans sa capacité à recréer l'objet, le sujet est fréquemment en proie à des épisodes dépressifs masqués par les passages à l'acte.

Balier\index{Balier} (1988) explique que le processus qui va amener la position dépressive est comme « un blocage de l'élaboration des imagos archaïques venant des premières relations avec l'objet, des manques dans la constitution de l'auto-érotisme et de la séparation soi-objet. C'est à ce niveau qu'il faut situer les échecs du processus de liaison des pulsions et l'action de la pulsion de mort, responsable de l'explosion de l'agressivité libre qui pourra se faire sentir à des moments d'organisation différents de la personnalité. » 

Nous pouvons dire que la violence, considérée comme une explosion de l'agressivité libre, résulterait d'un échec des liaisons pulsionnelles, qui trouverait son origine dans la relation avec le premier objet, notamment lors du processus de séparation soi-objet. C'est pourquoi il est intéressant de mettre en lien la clinique de l'acte violent avec le processus de séparation et individuation de l'adolescence, tout en comprenant sa relation primaire objectale. Cette déliaison fera agir la pulsion de mort sur la pulsion d'autoconservation, dans le but de préserver le narcissisme fragilisé du sujet par la défaillance du premier lien objectal.

Klein\index{Klein} va également expliquer la violence comme un défaut des premiers liens avec l'objet. 

\subsection{La violence comme résultante de la défaillance du Surmoi}

M. Klein\index{Klein} (1946) explique le comportement violent par une défaillance du Surmoi.

L'agressivité s'exprime par les pulsions destructrices, le sadisme et la haine qui vont représenter une partie de la pulsion de mort. Pour elle, l'enfant lutte très tôt contre ses propres pulsions agressives projetées vers le monde extérieur.

L'acquisition d'un Surmoi chez Klein\index{Klein} est très précoce car elle correspond à l'incorporation des premiers objets lors de la position schizo-paranoïde. Période pendant laquelle l'enfant ne distingue pas la différence entre sa mère et lui. Il appréhende le monde extérieur en introjectant le bon (positif pour le moi) et en rejetant le mauvais (négatif pour le Moi). L'angoisse du nourrisson est paranoïde car il a peur de la destruction du Moi par les objets persécuteurs. Lors de la position dépressive, apparaît la fusion progressive des pulsions agressives et libidinales par la réunion des objets partiels. La relation à l'objet n'est plus partielle mais totale. Il y a également la différenciation entre le Moi et le non Moi. Le même objet peut être considéré à présent comme bon mais aussi comme mauvais. L'angoisse de culpabilité liée au sentiment d'avoir agressé la personne aimée va orienter le sujet dans une phase de réparation. La tendance criminelle serait justement liée à l'échec des mécanismes de réparation. Klein\index{Klein} admet l'hypothèse d'une relation entre criminalité et psychose. Les actes criminels apparaissent ainsi sous l'aspect paradoxal d'être essentiellement défensifs, à l'encontre d'un entourage projectivement construit par le sujet comme menaçant et destructeur. C'est ce que nous pouvons mettre en relation avec le comportement violent de l'adolescent, qui va défendre son narcissisme contre un environnement qu'il va construire projectivement. Cependant, cette mise en acte envers l'environnement, n'aura pas la même signification selon le passage à l'acte ou selon l'acting-out.

\section{La mise en acte : Passage à l'Acte et « Acting-Out »}

L'adolescent va donc mettre en acte sa souffrance par le comportement violent et celle-ci va être dirigée contre un objet. Forget\index{Forget} (2005) va tenter de comprendre le processus de la mise en acte en différenciant passage à l'acte et acting-out. Pour lui, un acte est un franchissement qui engage le sujet dans une affirmation, une orientation, un choix. Le sujet passe par plusieurs étapes. Tout d'abord, un moment d'incertitude qui témoigne de ce qui l'anime, en s'appuyant sur des marques inconscientes en lui-même qu'il ignore. Puis le temps du choix et de l'agir. Enfin celui de l'après-coup où le sujet se retrouve ou non dans son choix, qui lui permet de se rendre compte des marques inconscientes qui le gouvernent à son insu. Les mises en acte, elles, font référence à une démarche réflexive très réduite voire impossible de la part de l'adolescent, ce qui va léser le rapport à l'autre, lors d'un passage à l'acte ou d'un « Acting-out ».

Laplanche\index{Laplanche} et Pontalis\index{Pontalis} (1967) définissent le passage à l'acte comme un cas particulier d'acting-out et ils ne voient pas la différence essentielle entre ces deux expressions. « L'acting-out » a été nommé par Freud\index{Freud} à partir de la cure analytique. Il désigne un passage à l'acte, un acte ou une action de caractère impulsif en rupture avec les motivations habituelles du sujet mais sous la forme auto ou hétéro-agressive, de ce qui n'a pas pu se dire dans le transfert envers l'analyste.  Quant à Lacan\index{Lacan} (1962), il distingue l'acting-out du passage à l'acte : selon qu'il s'agit de l'un ou de l'autre cas, la position du sujet est pour lui structurellement différente, et c'est cette différence de position qui va permettre de définir la mise en acte que nous allons étudier durant ce mémoire.

\subsection{Le Passage à l'acte}

Le passage à l'acte recouvre habituellement un acte agressif et violent au caractère délictueux ou / et impulsif: le viol, le vol et la violence en sont des exemples. Kammerer\index{Kammerer} (2000) définit le passage à l'acte comme résultant d'une confrontation avec un événement insurmontable pour l'appareil psychique ou encore de la rencontre traumatique entre un fantasme intrapsychique et une réalité externe qui semble lui donner raison, qui semble réaliser ce même fantasme. La solution violente s'impose au sujet, elle est le fait de fonctionner sur des défenses mises en place dans des expériences pervertissantes ou traumatiques très antérieures. 

Pour Forget\index{Forget} (2005), le passage à l'acte indique une remise en question de la place que l'adolescent occupe en tant que sujet. Il s'éjecte donc de cette place où la parole et la manifestation du « je » ne lui est pas possible et où l'Autre le contraint. Pour qu'il puisse dire « je », il faut que sa place soit reconnue, légitimée, afin de pouvoir manifester sa singularité. Ceci est une condition nécessaire à l'établissement de l'assise narcissique de sa subjectivité. Cette place peut-être activement  ménagée par un discours qui le concerne, par une ouverture en ce qui concerne son désir, afin qu'il puisse supposer légitime d'être reconnu, que sa parole soit prise en compte. S'il n'y a pas d'ouverture concernant son propre désir dans le discours de l'autre le concernant, il restera tributaire d'un discours qui le réduira à être l'objet de l'autre. Le statut d'objet réel d'autrui est insupportable au sujet. Il s'éjecte ainsi de cette place en laissant surgir ainsi le vide d'une inscription possible. 

En fait, dans le passage à l'acte, la jouissance se constitue dans le rapport de l'adolescent à l'autre en tant qu'objet. C'est le réel de cette perte qui offre une légitimité à la parole et qui génère le désir. Le passage à l'acte est finalement une tentative d'introduire dans le réel la référence à une perte, alors que la référence au réel de l'objet perdu fait défaut. De plus, il apparaît que le passage à l'acte est lié à un débordement d'angoisse, et celui-ci permettrait alors de « se libérer, de tenter de résoudre un conflit irrésolvable ». (Millaud\index{Millaud}, 1998) De ce fait, le passage à l'acte est un processus de décharge de tensions internes (Millaud\index{Millaud}, 1991,2002). Cette décharge est sans doute liée au processus de symbolisation qui est inexistant dans la psyché des sujets auteurs de ce type d'acte (Blos\index{Blos}, 2002).

\subsection{l' « Acting-out »}

Lacan\index{Lacan} (1962) pose l'acting-out comme un appel à l'Autre par son aspect démonstratif. L'acting-out se positionne donc du coté de la névrose : un surmoi tyrannique qui culpabilise. Il est structuré comme un scénario fantasmatique : Une mise en scène du refoulé, qui veut se placer en tant que messager orienté vers l'autre, chargé d'une forte agressivité qui se retourne contre le moi. Forget\index{Forget} explique l'acting-out, selon une orientation psychanalytique lacanienne, comme agi  par  un adolescent qui ne se rend pas compte de ce qu'il dévoile de lui-même. C'est par cette mise en acte qu'il manifeste sa subjectivité en souffrance. Ce que le sujet ne peut pas dire, il le met en scène. Cette mise en acte implique une instance symbolique représentant un spectateur ou un observateur, mobilisé mais piégé par l'écran de l'imaginaire. Il faut réintroduire la dimension symbolique exclue de la mise en place d'un transfert : L'adolescent face à ses interlocuteurs structurés dans leur discours par un signifiant exclu, qui lui permet d'être représenté lui-même comme signifiant. À l'adolescence, le type de mise en acte met à l'oeuvre le rapport du sujet à l'Autre dans les trois registres : réel, imaginaire et symbolique. La mise en scène est le défaut de la structuration symbolique de l'Autre. La mise en acte est de ce fait le résultat d'une défaillance de l'interlocuteur de l'adolescent dans sa structure symbolique ou dans sa fonction symbolique qu'il est censé exercer à l'égard du sujet. La sollicitation de l'interlocuteur se fait dans l'impasse de l'imaginaire. L'acting-out semble constituer une demande d'aide. Le passage à l'acte n'implique pas une demande d'aide adressée à l'autre. Lacan\index{Lacan} (1962) affirme que le passage à l'acte est un agir qui vient à la place d'un défaut de symbolisation. Ainsi au sein de l'institution, l'adolescent qui va recourir au passage à l'acte et celui qui va poser son acte en tant qu' « acting-out » n'aura pas le même rapport à l'institution. Tandis que l'un, par le passage à l'acte, tentera de se libérer des ses angoisses débordantes, cherchera une solution afin de se défaire de cette position d'objet, se réapproprier une position en tant que sujet, l'autre utilisera « l'acting-out » comme un appel à l'autre, qui est inconsciemment placé en tant que témoin de la manifestation de sa souffrance.

\vspace{1cm}
\centerline{*}
\vspace{1cm}

La violence est en fait en tout individu, car elle est fondamentale pour l'instinct d'auto-conservation. Cependant, nous pouvons comprendre que c'est la faillite du dualisme pulsionnel qui va expliquer le passage à l'acte violent. Cette désintrication pulsionnelle, peut s'expliquer par la perte de l'objet au moment où l'enfant parvient à unir ses pulsions instinctuelles libidinales et ses pulsions agressives. Ceci va avoir un impact négatif sur son Moi du fait de la survenue d'une faille narcissique  que le sujet tentera de compenser par la violence afin de tenter de se préserver. Cette fragilité du Moi qu'il va tenter de renforcer par des identifications, pourra plonger le sujet dans la dépression si celles-ci ne permettent pas de retrouver de défenses narcissiques solides. 

Le passage à l'acte est également décrit comme un moyen pour le sujet de se libérer des angoisses débordantes qu'il n'a pas su contenir. Bien que l'acting-out de l'adolescent soit travaillé au sein de la thérapie institutionnelle, le passage à l'acte nous intéresse davantage par sa recherche de contenance symbolique, fonction de l'institution que nous allons exposer par la suite.  Le passage à l'acte est une voie courte qui évite le détour par la psyché, et qui limite les possibilités d'élaboration mentale. Dans cette approche, on envisage surtout l'idée de défaillances  dans les capacités de contenance. (Raoult\index{Raoult}, 2006)  L'environnement, comme nous l'avons montré antérieurement avec la théorie de Winnicott\index{Winnicott}, va donc jouer un rôle extrêmement significatif sur cette contenance psychique chez le sujet violent, notamment lors de l'adolescence. Par le passage à l'acte violent, l'adolescent va mettre à l'épreuve un cadre institutionnel qui va jouer le rôle de pare-excitation, de contenance et d'étayage psychique.

\chapter{Clinique de l'institution}

\section{D'une clinique individuelle à une clinique sociale}

Freud\index{Freud} (1921) expose l'importance de l'articulation de la clinique individuelle avec le social : « L'opposition entre la psychologie individuelle et la psychologie sociale, ou psychologie des foules, qui peut bien nous paraître à première vue très importante, perd beaucoup de son acuité si on l'examine à fond (…). La recherche psychanalytique nous a appris que toutes ces tendances sont l'expression des mêmes motions pulsionnelles qui, dans les relations entre les sexes, poussent à l'union sexuelle, et qui, dans d'autres cas, sont certes détournées de ce but sexuel ou empêchées de l'atteindre, mais qui n'en conservent pas moins assez de leur nature originelle pour garder une identité bien reconnaissable. » D'ailleurs, Vallon\index{Vallon} (2004) se pose la question du rapport de la psychanalyse aux  mécanismes institutionnels. Freud\index{Freud} (1921) conçoit une continuité entre l'individu et l'autre : « Dans la vie psychique de l'individu, l'autre entre en ligne de compte très régulièrement comme modèle, comme objet, comme aide et comme adversaire, et de ce fait, la psychologie individuelle est aussi d'emblée, simultanément psychologie sociale. » C'est ainsi que Vallon\index{Vallon} définit l'institution comme une figure de l'Autre, aimable ou détestable. Cette figure de l'Autre, aura des répercussions sur le fonctionnement mental du sujet. D'ailleurs, Jeammet\index{Jeammet} (1985, 1992, 1994) explique qu'il existe un jeu de miroir entre le fonctionnement mental du sujet violent et l'institution. La violence provoque l'institution, la déstabilise et met à l'épreuve sa capacité à se maintenir et à être thérapeutique. Mais l'institution est elle, elle-même génératrice de violence ? Le sujet violent révèle les failles de l'institution, mais les dysfonctionnements de celle-ci mobilisent les potentialités de violence des patients. Ce jeu de miroir  et ces effets de résonance témoignent de l'extrême sensibilité de ces sujets aux variations de l'environnement et par là même des difficultés de leur appareil psychique à jouer un rôle tampon de médiation et d'élaboration. C'est l'entourage qui devient le lieu où se traite ce que l'appareil psychique ne peut pas traiter. Il va donc être un moyen de figuration de fonctions psychiques normalement dévolues au fonctionnement mental. Ce jeu dialectique entre le monde psychique interne, celui des représentations et des symboles, et la réalité externe est au centre de la problématique de l'agir et de la violence. 

L'institution endosse donc le rôle de figuration psychique pour l'adolescent violent en déficit de figuration mentale. Il va alors provoquer l'institution et interroger sa capacité à être thérapeutique. « Se situant au carrefour du dedans et du dehors, balisant les rapports du singulier et du pluriel, de l'intra-, de l'inter-personnel et du transpersonnel, l'institution est une instance d'articulation de formations psychiques extrêmement sensibles aux effets de la déliaison. » (Pinel\index{Pinel}, 1996) 

Marty\index{Marty} (2007) affirme que l'institution va se constituer en tant que pare-excitation si elle est capable de contenir l'excitation pulsionnelle, de lui donner forme et limite, puis de lui donner sens. Comme nous venons de l'expliquer, l'institution sera dans un premier temps, une instance d'articulation psychique pour l'adolescent en carence de fonction symbolique. L'institution jouera pour lui le rôle d'une figuration symbolique. Mais si l'institution est perçue par le sujet violent comme représentant un espace psychique en miroir du sien, comment pourra-t-elle pour lui se constituer en pare-excitation ? Pour Balier\index{Balier} (1988), il faut restaurer la fonction de pare-excitations en prenant soin des conduites thérapeutiques. Il n'est pas possible de la situer par rapport à une fonction spécifique. Nous allons observer comment l'institution va établir cette prise en charge thérapeutique qu'est la fonction de pare-excitation.

\section{L'institution comme cadre thérapeutique participant à la fonction de pare-excitation}

Pour l'adolescent en carence de fonction  symbolique  l'institution participe  donc d'un espace psychique interne.
La première institution du patient est sa famille, (Marty\index{Marty}, 2002), pas seulement ses parents et sa fratrie, mais aussi et surtout ses parents oedipiens, les images parentales qu'il a pu ou non intérioriser, avec et contre lesquelles il va se construire. Cette institution familiale est l'espace psychique que se construit l'adolescent pour penser sa conflictualité oedipienne. L'institution soignante va concourir à l'établissement de cette conflictualité interne nécessaire pour qu'il élabore la violence qu'il porte en rencontrant, en lui, les fantasmes qui le persécutent. Nous avons vu que le narcissisme déstabilisé d'un adolescent est d'autant plus fragilisé par un échec relationnel avec l'objet primaire qu'il n'a pas pu élaborer symboliquement. Comme le précise Balier\index{Balier} (1988), cette décharge de l'agressivité primaire devenue libre va chercher une sorte de contenant qui empêchera la désorganisation psychique dont le sujet est menacé. Dans le cas d'un adolescent violent, l'idée de contenance va jouer un rôle majeur dans la clinique de l'institution. Le contenant est selon Anzieu\index{Anzieu} (1985), une particularité du « Moi-Peau ». Ce Moi-Peau, considéré originellement dans la dimension du sujet, va posséder plusieurs spécificités que nous allons projeter vers l'instance collective que constitue l'institution, comme la maintenance, la contenance avec un « contenu » et un  « conteneur » . Ces spécificités vont servir à définir la fonction de pare-excitation.

\subsection{L'institution : « une maintenance et contenance psychique » : L'importance de son principe d'intervention}

La fonction de contenance de l'institution permet de traiter la violence de la pulsion chez ces adolescents pour lesquels cette fonction a été défaillante ou inexistante depuis l'enfance. Effectivement, comme nous l'avons vu, les pulsions et fantasmes destructeurs de l'enfant envers  l'« objet-mère » vont tester la permanence et la résistance qui prend progressivement valeur d'objet détruit/créé et d'objet progressivement objectalisé, différencié de la mère. Ceci va pouvoir engager le sujet vers la séparation et la réparation progressive, par la pensée, de ce qu'il fallait détruire pour exister : ces prémices sont indispensables pour mettre en place la fonction symbolique reconstituant ce qui a été préalablement détruit.

Reprenant la théorie de Anzieu\index{Anzieu} (1985), ce Moi-peau remplit tout d'abord, une fonction de maintenance du psychisme. La fonction biologique est exercée par ce que Winnicott\index{Winnicott} (1951) a appelé le « holding », c'est-à-dire par la façon dont la mère soutient le corps du bébé. La fonction psychique se développe par intériorisation du holding maternel. Le Moi-peau est une partie de la mère qui a été intériorisée et qui maintient le psychisme en état de fonctionner, du moins pendant la veille, tout comme la mère maintient en ce même temps le corps du bébé dans un état d'unité et de solidité. Il va ensuite remplir une fonction de contenance, fonction exercée par le handling maternel qui inclut tous les soins appropriés aux besoins de l'enfant par la mère. Kaës\index{Kaës} (1988) distingue deux aspects de cette fonction : « le contenant » qui est stable et immobile qui s'offre en réceptacle des sensations-images-affects du bébé, et le « conteneur » qui correspond à l'aspect actif qui transmet ses sensations-images-affects rendues présentables.

Dans l'institution, nous pouvons émettre l'idée que la contenance est symbolisée par le cadre institutionnel, avec toutes ses lois et ses règles. Celle-ci peut être symbolisée par le soin éducatif, pédagogique et thérapeutique de l'institution, avec comme contenant l'institution en elle-même, et comme conteneur, tous les membres professionnels actifs qui se présentent au sein de cette institution. Comment l'institution va t-elle intervenir pour symboliser ces spécificités ?

\subsection{L'importance d'un « cadre »}

Toutes les institutions possèdent un projet d'établissement ainsi qu'un cadre de règles. Achaintre (1992) explique justement que tout groupe ne peut se passer d'un cadre et de règles dites ou non-dites. On trouve, au minimum une unité de lieu, de temps et d'action et un ou des buts. Le groupe, s'il veut durer, devient une institution lorsqu'il est numériquement important et s'il organise un lieu de vie. Achaintre interroge la définition de l'institution et y répond en la décrivant comme une organisation humaine qui possède une finalité et qui présente les critères sociologiques d'une organisation : existence d'un objectif, division des tâches et des rôles, division de l'autorité, hiérarchie, système de communication, contrôle des résultats et de l'orientation générale. Il poursuit sa définition en soulignant les problèmes posés par les institutions du fait de leur visée unifiante et leurs tensions permanentes. Pour lui, il faut avant tout repérer ces tensions à terme bipolaires : la visée administrative et la visée soignante ; l'abord du patient à partir du processus organique et du processus mental (biologique et psychologique) ; la face publique et la face intime (social et psychologique) ; le contenu psychique manifeste et le contenu latent ; la nécessité d'avoir un milieu accueillant, permissif, et celle d'avoir un milieu contenant, limitant ; ou encore le projet thérapeutique individuel  et le projet thérapeutique groupal ; le parler ou l'agir.

Pinel\index{Pinel} (1996) va proposer plusieurs principes méthodologiques d'intervention. « La situation d'asymétrie » qui transmet la loi de l'offre et de la demande, prédétermine un travail de mise en formes des attentes, des pré-investissements, des pré-représentations et des projections qui vont prédéterminer les conditions d'une relation de type transfert/ contretransfert. Ce travail de mise en forme des attentes va être maitrisé par la mise en place d'un « cadre ». Celui-ci établit un système d'invariants de temps, de lieu et d'action. Ce processus se développe à l'intérieur de ce cadre, il est déterminé par l'appareillage psychique des intervenants et par l'énoncé d'une règle appropriée à la situation. L'instauration de « l'objet et des objectifs » permettra de comprendre « ce qui de la réalité psychique est mobilisé, étayé et immobilisé par l'institution. » (Kaës\index{Kaës}, 1996). Pour finir, la mise en place « d'opérateurs » acteurs des  conditions de l'écoute et les dispositifs de traitement du matériel est essentielle. Cette mise  en place de ces principes de traitement va participer au processus de maintenance et de contenance psychique.

Lebrun\index{Lebrun} (2008) poursuit cette définition de l'institution en soulignant l'importance de la  signification dans la définition de l'institution de l'action d'instituer. C'est-à-dire le travail de fondation  de la chose instituée qui englobe le groupe  et le social. Il poursuit en soulignant l'importance de la transmission de l'institution par la tradition. L'institution prend son assise dans le passé, pour pouvoir être remanié régulièrement par la suite. Pour Tosquelles\index{Tosquelles} (1973), un travail était nécessaire pour replacer l'institution en tant qu'institution, et ne pas réduire l'institution en tant qu'établissement, lieu où les échanges restait figé par la tradition. 

Toutes ces données participent à l'instauration d'un cadre solide réalisé par une équipe pluridisciplinaire afin de préserver cette fonction de maintenance et de contenance. Pour Legendre\index{Legendre} (2007), la fonction institutionnelle est « le noyau de la civilisation où s'organise le lien subjectif et social ». 

\subsection{L'institution : une représentation symbolique de la société et de la culture}

L'institution va donc représenter un environnement respectant les codes de la société. Pour Kaës\index{Kaës} (1986) l'institution est d'abord une formation de la société et de la culture. Elle est l'ensemble des formes et des structures sociales instituées par la loi et la coutume : l'institution règle nos rapports, elle nous préexiste et s'impose à nous. Chaque institution est dotée d'une finalité qui l'identifie et la distingue. Une première distinction est importante à établir entre l'opposition et l'articulation de l'instituant et de l'institué. Cette opposition fait référence au symbolique. En effet, l'imaginaire, est la capacité originale de production et de mise en œuvre des symboles qui sont liés à l'histoire et évoluent. L'imaginaire est l'attribution de significations nouvelles à des symboles déjà existants. L'imaginaire individuel « préexiste et préside à toute organisation, même la plus primitive de la pulsion. C'est à un fonds de représentations imaginaires que la pulsion emprunte, « au départ » sa « délégation par représentation ». L'imaginaire social est, avec la nécessité de l'organisation et des fonctions, à la source de l'institution.

La seconde distinction oppose et articule institution et organisation. L'organisation aurait un caractère concret qui disposerait moins des finalités que des moyens de les atteindre.

Ces deux distinctions sont nécessaires à la définition de l'institution car ces processus vont permettre d'articuler des fonctions psychiques, comme le retournement de la finalité institutionnelle, avec celle du retournement psychique. Ainsi l'institution, qui est une formation sociale et culturelle, va également permettre la réalisation des fonctions psychiques pour les sujets dans leur structure, leur dynamique et leur économie personnelle. Elle mobilise des investissements et des représentations qui contribuent à la régulation endopsychique et qui assure les bases de l'identification du sujet à l'ensemble social ; selon Kaës\index{Kaës} (1986), elle constitue l'arrière-fond de la vie psychique dans lequel peuvent être déposées et contenues certaines des parties de la psyché qui échappent à la réalité psychique.

Freud\index{Freud} (1921) va admettre que c'est l'institution qui est la donnée primaire de l'identification et de la formation du Moi. Son analyse est fondée sur les rapports entre les identifications et la formation du moi, sur l'étude des deux institutions fondamentales tels que l'Église et l'Armée. Il n'analyse pas l'Armée ou l'Église en soi, mais la forme permanente que prend pour l'inconscient l'Armée ou l'Église.

Bergeret\index{Bergeret} (1988) affirme que la psychothérapie institutionnelle joue sur les limites du Moi par une action de contenant signifiant direct, ou de contenants de signifiants symboliques (les lois, les règlements, en tant que tels et tout ce que ça signifie ; le temps des horloges et l'espace géographique et tout ce que l'on peut organiser). L'institution est un élément de la culture, c'est-à-dire une réalité mentale collective ayant des aspects conscients mais aussi des aspects inconscients. Ces aspects inconscients sont créateurs d'illusions. L'illusion de créer une société idéale, l'illusion que l'équipe est là pour faire la psychothérapie de ses membres, l'illusion que l'institution est un divan et l'illusion de l'état fusionnel. Ces illusions, si elles sont repérées, peuvent être l'objet d'un traitement et devenir source du progrès.

Avec son principe d'intervention organisé, sa formation sociale et culturelle obéissant à des règles qui lui sont propres et qui vont participer à l'élaboration d'un cadre solide, l'institution se réalisera symboliquement comme « appareil psychique groupal » (Kaës\index{Kaës}, 1970).  Cet appareil articulera un espace de symbolisation qui va accueillir, gérer et transformer les éléments pulsionnels qui immobiliseront les formations psychiques communes (Pinel\index{Pinel},1996). Il faut que ce cadre résiste à la tension qui s'exerce sur lui pour qu'il continue de remplir sa fonction de garant d'une certaine permanence de la forme. L'institution peut être considérée comme un espace contenant la dynamique psychique des membres qui la composent. 

Le Moi-peau vise à envelopper tout l'appareil psychique. Le Moi-peau est alors figuré comme écorce, le ça pulsionnel comme noyau, chacun des deux termes ayant besoin de l'autre. Le Moi-peau n'est contenant que si il a des pulsions à contenir. La pulsion n'est ressentie comme poussée, comme force motrice, que si elle rencontre des limites et des points spécifiques d'insertion dans l'espace mental où elle se déploie et que si sa source est projetée dans des régions du corps dotées d'une excitabilité particulière. Cette complémentarité de l'écorce et du noyau fonde le sentiment de la continuité du Soi.

À la carence de cette fonction conteneur du Moi-Peau, peut répondre à une angoisse, celle d'une excitation pulsionnelle diffuse, permanente, éparse, non localisable, non identifiable, non apaisable, qui va traduire une topographie psychique sans écorce.

Nous pouvons mettre en lien la carence de cette fonction du Moi-peau avec le concept de « deprivation »  de Winnicott\index{Winnicott} (1956), moment de la perte de l'objet primaire au moment de l'union de ses pulsions. Si son agressivité primaire projeté vers l'objet pour permettre de se positionner en tant que sujet et moins en tant qu'objet, (ce qui lui permettra d'élaborer son narcissisme) ne va pas être contenue par l'environnement familial, l'enfant basculera dans la violence, car il n'aura pas trouvé de limite à son expression. L'institution qui accueillera l'adolescent violent pourra alors jouer le rôle d'un moi-peau psychique, qui contiendra ses pulsions psychiques, qui rencontreront alors à ce moment-là des limites.

Freud\index{Freud} (1895) a reconnu au Moi, une fonction de pare-excitation. Il reconnaît que la mère sert de pare-excitation auxiliaire au bébé et Anzieu\index{Anzieu} (1985) ajoute : jusqu'à ce que le Moi en croissance de celui-ci trouve sur sa propre peau un étayage suffisant pour assumer cette fonction.

\subsection{L'institution comme fonction de « pare-excitation~»}

Le terme de « pare-excitation » est employé par Freud\index{Freud} dans le cadre d'un modèle psychophysiologique pour désigner une certaine fonction, et l'appareil qui en est le support. La fonction consiste à protéger l'organisme contre les excitations en provenance du monde extérieur qui, par leur intensité, risqueraient de le détruire. L'appareil est conçu comme une couche superficielle enveloppant l'organisme et filtrant passivement les excitations.

Balier\index{Balier} (1988) souligne la nécessité de la restauration d'un pare-excitations au sein des cadres pénitentiaires accueillant des patients présentant un régime pulsionnel débordant. En effet, selon lui, il faut faire face aux grandes quantités d'énergie qui devront se transformer en petites quantités pour pouvoir permettre l'instauration de la pensée. C'est pour cela qu'il justifie le lieu qui n'est pas uniquement vécu comme un lieu de souffrance, mais aussi un lieu où les patients se sentiront apaisé et protégé des sollicitations impossibles à fuir. Pour lui, l'intériorisation du cadre pénitentiaire constitue une partie de pare-excitation. Ce cadre est régit par un ensemble d'interdits qu'on peut qualifier de « Surmoi rudimentaire ». L'intériorisation de ce cadre concerne également l'intégration du cadre maternel. Celui-ci est renforcé par l'identification au cadre soignant. Le patient s'identifie au soignant qui cherche à comprendre ce qui se passe à l'intérieur de lui en replaçant les divers éléments perçus au sein du déroulement d'une histoire personnelle. C'est par là que va s'instaurer le travail préconscient et le travail de liaison généré par la volonté de comprendre ce qui se passe en lui-même, et pour E. Kestemberg\index{Kestemberg}, cette dynamique constitue la fonction de pare-excitation.

Ainsi, en s'offrant comme espace externalisé pouvant accueillir le monde interne du sujet, l'institution devient une seconde peau psychique sur laquelle viennent s'inscrire les éprouvés quotidiennement vécus (Marty\index{Marty}, 2007). Le travail institutionnel consiste à donner de la profondeur à ces éprouvés, à les contenir et les relier entre eux jusqu'à ce qu'ils constituent la trame d'une histoire vivante faite d'affects où la discontinuité apparente de ces diverses expériences finit par se fondre dans une continuité de vie qui fasse sens. La peau institutionnelle se retourne comme une enveloppe : d'abord surface d'inscription, elle se transforme en contenant à l'intérieur duquel vont se rejouer, dans la répétition transférentielle, les mouvements de la vie psychique du patient. Mouvement qui nous rappelle la distinction contenant/conteneur que nous propose Anzieu\index{Anzieu} (1993). L'institution et ses professionnels s'offrent comme un espace de projection. Cette transformation du rapport que le sujet établit avec lui-même n'est possible que parce-que l'institution est une surface d'inscription sur laquelle se retiennent les mouvements psychiques, au delà du déni. L'institution a une fonction psychique permettant la secondarisation, le déplacement sur d'autres objets, autorisant une réflexion au double sens d'un renvoi d'une image et d'une saisie de cette image, de son intériorisation. 

À l'image d'une peau psychique qui enveloppe, contient, mais s'adapte aux mouvements du corps pour que jamais elle ne casse, le cadre thérapeutique doit alors jouer ce rôle d'enveloppe psychique qui tient, contient et s'adapte sans rompre. Ces qualités du cadre (sa fluidité, sa souplesse dans la fermeté) peuvent se développer dans la mesure où l'institution prend soin d'elle même, c'est-à-dire dans la mesure où elle analyse les mouvements qui se déploient en son sein. 

L'institution peut donc être pensée comme cadre, enveloppe et contenant (Bion\index{Bion},1963). Elle est un lieu psychique où viennent se déposer les traces laissées par les patients dans leurs mouvements transférentiels et par les soignants dans leurs mouvements contre-transférentiels sur le cadre de travail qu'elle représente. La relation soigné/ soignant est ainsi toujours contextualisée par une autre relation qui agit sur une autre scène, le plus souvent à l'insu des protagonistes : l'infantile s'exprime dans un jeu relationnel qui aura l'institution comme toile de fond (Marty\index{Marty}, 2007). L'institution sera donc une structure intermédiaire entre l'appareil psychique de l'adolescent, et la prise en charge thérapeutique de celle-ci par son contenu (cadre, règles, lois) et conteneurs (équipe pluridisciplinaire).

Ceci va nous rappeler que le Moi-peau est une structure intermédiaire de l'appareil psychique : intermédiaire chronologiquement entre la mère et le tout-petit, intermédiaire structurellement entre l'inclusion mutuelle des psychismes dans l'organisation fusionnelle primitive et la différenciation des instances psychiques correspondant à la seconde topique freudienne. Sans les expériences adéquates au moment opportun, la structure n'est pas acquise, ou, plus généralement, se trouve altérée, d'où l'importance de cette figuration symbolique pour les institutions accueillant les adolescents ayant recours aux comportements violents.

De plus, le fonctionnement psychique conscient et inconscient a une part de lui qui vise à l'indépendance alors qu'il est, dès l'origine, doublement dépendant : du fonctionnement de l'organisme vivant qui lui sert de support, et des stimulations, des croyances, des normes, des investissements, des représentations émanant des groupes dont il fait partie (à commencer par la famille, à continuer par le milieu culturel) (Anzieu\index{Anzieu}, 1985).

Anzieu\index{Anzieu} va reprendre cette double dépendance avec Kaës\index{Kaës} (1988) en postulant l'existence d'un double-étayage du psychisme : sur le corps biologique, sur le corps social, et d'autre part, un étayage mutuel : la vie organique et la vie sociale ont l'une et l'autre autant besoin d'un appui quasi-constant sur le psychisme individuel que celui-ci a besoin d'un appui sur le corps vivant et sur un groupe social vivant.

\vspace{1cm}
\centerline{*}
\vspace{1cm}

Nous avons fait le constat que la dépendance de l'environnement de l'adolescent témoigne d'une fragilité intrapsychique (dû au remaniement psychique et narcissique de l'adolescence), qui ne lui permet pas de constituer seul sa propre identité. Nous avons également vu que l'environnement permet la renégociation du lien à l'objet. Cependant, lorsque certains sujets ont rencontré une défaillance lors de la relation primaire objectale, le processus de séparation et d'individuation de l'adolescence va se révéler plus complexe. Or, une relation primaire objectale de bonne qualité est nécessaire car elle permet l'accès à la fonction symbolique. N'ayant pas accès à la fonction symbolique qui lui permettrait de se représenter la séparation donc l'individuation, l'adolescent va trouver des solutions afin de pallier ce manque de symbolisation, par le passage à l'acte. Bien que l'acting-out de l'adolescent soit travaillé au sein de la thérapie institutionnelle, le passage à l'acte nous intéresse davantage par sa recherche de contenance symbolique, fonction de l'institution que nous venons d'exposer. Or, cette recherche de contenance symbolique peut être rapprochée de la relation primaire objectale du sujet qui permet l'accès aux fonctions symboliques. C'est ainsi que nous pouvons dire que le passage à l'acte violent de l'adolescent est une tentative d'accéder à ces fonctions symboliques. L'institution va pouvoir y répondre par sa fonction de contenance et de pare-excitation, tout comme l'objet-mère. L'infantile des sujets s'exprime dans un jeu relationnel qui aura l'institution comme toile de fond (Marty\index{Marty}, 2007). L'institution se constitue en tant que figuration psychique qui répondra à la demande de contenance de pare-excitation et de contenance en utilisant des moyens symboliques tel que son principe d'intervention. Mais pourquoi, pour le sujet adolescent dans le passage à l'acte violent, cette fonction de pare-excitation, ne semble pas fonctionner à certains moments ?


\part{Partie méthodologique}

\chapter{Problématique}

Notre objet de recherche est la compréhension de la mise en échec de la fonction de pare-excitation de l'institution par les sujets adolescents dans le passage à l'acte violent. Notre recherche va se baser sur la confrontation des apports théoriques entre la clinique de l'adolescent, la clinique de l'acte et la clinique de l'institution. Nous avons vu que l'adolescent souffrant de troubles du comportement violent présente une fragilité narcissique marquée à la fois par ses modifications pubertaires, d'après la psychopathologie de Marcelli\index{Marcelli} et Braconnier\index{Braconnier} (1992), mais également par un défaut de l'apprentissage des fonctions symboliques pendant l'enfance, comme le précise Winnicott\index{Winnicott} (1939). L'objet primaire n'aura pas permis l'élaboration des imagos archaïques, et de la séparation entre soi et objet (Balier\index{Balier}, 1988). Or, le passage à l'acte est un agir qui vient à la place d'un défaut de symbolisation, postulat émit par Lacan\index{Lacan} (1962). 

L'institution incarne le lieu de figuration psychique et se maintient en tant que pare-excitatrice, qui a pour fonction de contenir les débordements pulsionnels, par le soin de ses conduites thérapeutiques (Balier\index{Balier}, 1988). Cet espace interne psychique, (Marty\index{Marty}, 2007)joue le rôle de pare-excitation par étayage pour l'adolescent souffrant d'une fragilité narcissique en besoin de recourir à un objet externe qu'il va intérioriser Kaës\index{Kaës}, (1988). L'intériorisation du cadre par l'usager constitue une part de pare-excitation et peut-être considérée comme l'intégration du cadre maternel (Balier\index{Balier}, 1988).

Concernant notre population, il est important de souligner que les adolescents, se situent dans une période où leur Moi, (leur noyau narcissique) va être profondément déstabilisé. Ce sont  les aspects dynamiques, appartenant au processus de la puberté, qui sont responsables de ces répercussions (Marcelli\index{Marcelli} et Braconnier\index{Braconnier}, 1992). Plusieurs auteurs tels que Laufer (1964), Freud\index{Freud} (1914), Kestemberg\index{Kestemberg} (1962), Jeammet\index{Jeammet} (1989) et Winnicott\index{Winnicott} (1956), s'accordent à dire que le sujet tentera de protéger cette fragilité en s'appuyant massivement sur son environnement, ce qui lui permettra de réactualiser le processus de séparation et d'individuation (Blos\index{Blos}, 1967). Ce processus archaïque, qui lui a permis d'intérioriser l'objet primaire avant de pouvoir s'en différencier, lui permettra d'acquérir de bonnes assises narcissiques. Si celles-ci sont de bonne qualité, lors de l'adolescence, pendant la réactualisation de séparation et d'individuation par le rejet des identifications parentales antérieures, le sujet pourra quand même trouver des solutions bénéfiques.

Cependant, la population qui nous intéresse précisément dans ce mémoire, est le sujet adolescent dans le passage à l'acte violent. Chez lui, cette fragilité narcissique au moment de l'adolescence sera d'autant plus accentuée, car il n'aura pas acquis les bases narcissiques nécessaires pendant son enfance. Notre partie théorique témoigne que l'adolescent présentant des troubles du comportement violent a subi la perte d'un bon objet, au moment de son enfance où il tente de parvenir à l'union de ses pulsions instinctuelles, libidinales et agressives (Winnicott\index{Winnicott}, 1939). 

C'est dans le Moi que se construisent les échanges narcissiques mère-enfant. L'agressivité primaire, permettant de développer le vrai self, va rendre possible la construction d'un Moi, et la différenciation progressive mère-enfant. L'objet-mère, détruit puis recréé, permet à l'enfant de se différencier de la mère. La séparation et la réparation progressive par la pensée, mettent en place la fonction symbolique. Cet environnement va contenir cette agressivité. Si cet environnement n'y parvient pas, l'enfant ne trouvera pas de limites à son expression et plongera dans la violence. 

Balier\index{Balier} (1988) va préciser que c'est au niveau du blocage des imagos archaïques venant des premières relations avec l'objet, des manques dans la constitution de l'auto-érotisme et de la séparation soi-objet, qu'il faut replacer les échecs du processus de liaison des pulsions et l'action de la pulsion de mort, responsable de l'explosion de l'agressivité libre. L'adolescent utilisera la violence par la mise en acte pour tenter d'accéder à la fonction symbolique.

Du point de vue psychanalytique, Lacan\index{Lacan} (1962) affirme que le passage à l'acte est un agir qui vient à la place d'un défaut de symbolisation. Cette spécificité de la mise en acte qu'est le passage à l'acte va plus particulièrement nous intéresser dans ce mémoire. Nous allons y interroger ce défaut de symbolisation en lien avec l'institution  qui va incarner justement une instance psychique symbolique, processus de symbolisation qui est inexistant dans la psyché des sujets auteurs de ce type d'acte (Blos\index{Blos}, 2002).

Nous avons vu que l'institution remplit sa fonction pare-excitatrice par son principe d'intervention comprenant un cadre appuyé sur un règlement et par le soin de ses conduites thérapeutiques. L'institution s'articule comme un espace psychique interne (Marty\index{Marty}, 2007). L'adolescent, du fait de sa fragilité narcissique, l'investira, dans une fonction d'étayage, comme objet externe à intérioriser. Et c'est ainsi que l'institution pourra contenir les débordements pulsionnels de l'adolescent. Par cette fonction de pare-excitation, l'adolescent en défaut de symbolisation et souffrant d'une fragilité narcissique trouvera ainsi dans l'institution un renfort narcissique par la représentation d'une instance psychique symbolique dont il fait défaut. 

Cependant, cette fonction de pare-excitation est mise en échec par certains usagers. Alors une question s'impose à nous :

Si la fonction de pare-excitation de l'institution se révèle globalement efficiente, alors comment peut-on comprendre la mise en échec ponctuelle de celle-ci par le sujet adolescent inscrit dans le passage à l'acte violent ?

\paragraph{Hypothèse générale}

La mise en échec de la fonction de pare-excitation de l'institution peut être interprétée comme la résultante d'une tentative de symbolisation de la séparation et de l'individuation chez l'adolescent dans le passage à l'acte violent. Par sa fonction de pare-excitation, elle réactualise dans ce cas la relation primaire objectale qui n'a pas pu permettre la mise en place de la fonction symbolique.

\chapter{Protocole de recherche}

Cette hypothèse générale suppose alors d'évaluer le processus de symbolisation de la séparation et de l'individuation du sujet adolescent violent, son passage à l'acte violent et la mise en échec de la fonction de pare-excitation de l'institution.

Le processus de symbolisation de la séparation et de l'individuation se définit par la capacité du sujet à faire appel à ses propres ressources psychiques ou à une figuration psychique externe afin de permettre cette séparation. Nous allons opérationnaliser ce processus selon plusieurs variables. Tout d'abord, l'investigation sur la nature des liens du sujet avec sa famille et son parcours de vie, nous permettra de capter des informations concernant la nature des relations primaires du sujet, responsables de l'advenue des fonctions symboliques. De plus, l'analyse de sa relation à l'objet, ici l'institution, va nous permettre de saisir l'échec de ce processus de symbolisation, c'est-à-dire, comprendre si le sujet dépend massivement de l'objet et de son environnement ou bien si il est capable de se séparer et de s'individualiser seul. L'incapacité à symboliser du sujet par l'analyse de l'investissement relationnel à travers un appel à l'autre, ses capacités de pare-excitation, de contenance et d'étayage confirme la difficulté à fonctionner seul. 

Le passage à l'acte violent du sujet adolescent se définit par un agir violent envers l'institution, qu'il s'agisse de l'établissement ou d'un professionnel la représentant. Cette variable suppose d'évaluer le rapport de l'acte du sujet en lien avec l'institution, en comprenant alors la dynamique psychique du sujet, tel que le traitement de l'excitation, l'élaboration psychique, et sa relation à l'objet, donc dans ce cas, sa représentation de l'institution. Ces éléments permettront de comprendre la signification du passage à l'acte du sujet en lien avec son rapport à l'institution.

La mise en échec de la fonction de pare-excitation de l'institution se définit lorsque le sujet adolescent utilisera l'agir violent au sein de l'institution. Dans un premier temps, cette variable suppose d'évaluer la fonction de contenance donc de pare-excitation de l'institution chez le sujet, et de mettre en lien cette fonction avec son rapport à l'acte violent.

\section{Population}

Notre mémoire portant sur l'échec de la fonction de pare-excitation de l'Institution chez les sujets adolescents dans le passage à l'acte violent, nous avons rencontré plusieurs sujets au sein d'un ITEP (Institut Thérapeutique Éducatif et Pédagogique). Les adolescents pressentis devaient avoir eu recours au moins une fois à un acte violent ayant pour objet l'Institution, que ce soient  les professionnels  qui la représentent ou les biens matériels dont elle dispose. Nous avons ainsi retenus deux sujets de 15 et 14 ans.

Nos entretiens de recherche se sont déroulés au sein de l'ITEP de Grèzes, dans le département de l'Aveyron, après avoir obtenu l'autorisation de l'institution. C'est en accord avec Mme Hermet, psychologue clinicienne, que nos entretiens ont pu être réalisés. Elle nous a proposé une liste de plusieurs jeunes qui correspondaient à nos critères de recherche afin de les rencontrer. Cette première rencontre avec ceux-ci aura pour objet  de  se présenter, d'expliciter notre recherche ainsi que les principes déontologiques sur lesquels elle se fonde, et d'obtenir leur accord. Nous avons décidé de ne pas réaliser l'entretien de recherche le même jour que notre présentation  afin de   laisser le temps de la réflexion aux adolescents pour leur offrir la possibilité de se rétracter. Cette approche permettait de respecter la volonté de participation du sujet en limitant le risque de l'intimidation induite par la différence de statut entre adulte et adolescent.

Afin de respecter l'anonymat des sujets, leurs prénoms ont été modifiés.
            
\section{Méthode}

Un entretien semi-directif comportant des questions sur trois dimensions, comprenant celle du sujet adolescent, celle de son rapport à l'acte, et pour finir celle de son rapport à l'institution, aidera à saisir des données dont nous avons exposé notre opérationnalisation dans notre protocole de recherche. L'ordre des questions n'est pas immuable  mais respectera notre protocole de recherche.

De plus, nous avons choisi d'administrer le test de Rorschach comme recueil de données complémentaires d'un entretien semi-directif. L'association de ces deux moyens de recueil de données  favorisera l'élaboration des modalités que nous recherchons. Un entretien semi-directif isolé pourrait compromettre la fiabilité des résultats car l'établissement d'une relation duelle peut-être ressenti de façon angoissante pour l'adolescent. L'utilisation d'un test projectif offre une médiation qui pourra tempérer les effets de cette relation duelle, (Jeammet\index{Jeammet}, 1993) et appuyer et compléter les réponses obtenues grâce à l'entretien semi-directif. De plus, (Moulin\index{Moulin}, 2010) le sujet pourrait-être réticent à se livrer et le Rorschach permettrait de médiatiser cette rencontre.

\subsection{Passation des entretiens}

Nous avons contacté plusieurs adolescents de l'ITEP, afin de nous présenter ainsi que notre recherche. Nous leur avons expliqué le déroulement de l'entretien de recherche, ainsi que leur contenu tout en soulignant la confidentialité de celui-ci. À l'issue de cette rencontre, l'adolescent avait le choix d'accepter ou de refuser. La passation a duré en moyenne 2h et s'est déroulé en deux entrevues, dans un premier temps par la passation de l'entretien semi-directif et pour terminer par la passation du test de Rorschach.

Le questionnaire n'a pas été forcément suivi dans l'ordre prévu afin de ne pas rompre le discours du sujet par une rigidité qui aurait pu déranger la fluidité de ses propos mais nous avons veillé à répondre à toutes les questions de notre recherche. Il est important de demeurer habile et à l'écoute sans imposer des questions mécaniques qui auraient pu être interprétées par le sujet  comme intrusives. Il est primordial de mettre à l'aise le sujet, par une neutralité bienveillante, notamment chez les adolescents, qui auraient pu être impressionnés et mal à l'aise lors de notre passation. Leur laisser le choix de la parole, tout en les aidant avec des questions et des relances, nous paraît plus bénéfique qu'un recueil de données trop rigide et directif.

La consigne précédant l'entretien a été la suivante : «  Je vais te poser des questions, sur toi, sur l'institution, comment tu te sens ici, à quel moment ça ne va pas. Si il y a des questions auxquelles tu n'as pas envie de répondre, tu as le droit de ne pas y répondre. Tout ce que tu vas me dire va rester confidentiel. »

Les entretiens se sont déroulés dans une salle où se présentait un bureau massif  avec un fauteuil derrière et une chaise en face. À l'extrémité de la salle, se situait une table ronde avec plusieurs chaises, et c'est autour de cette table ronde que nous avons fait le choix de la passation, afin de ne pas se positionner de face par rapport au sujet qui aurait pu interpréter nos intentions d'intrusives. La table ronde nous situait de biais par rapport au sujet, en laissant une chaise inoccupée avec le sujet, afin d'assurer une bonne distance pour éviter qu'il se sente « oppressé ». 

\subsection{Analyse des entretiens}

La première partie de cet entretien, visait à recueillir des données concernant le parcours de vie du sujet adolescent. Cette recherche devait permettre de saisir la place de sujet en tant qu'adolescent tout en considérant la dynamique psychique propre de cette période de la vie. Cette investigation avait pour objet de saisir la relation du testé avec les objets parentaux, l'objet primaire, mais également des éléments périphériques concernant son environnement familial et affectif. De plus, elle devait nous apporter  des informations sur son mode relationnel avec ses pairs, ses activités, de manière à repérer ses nouveaux supports identificatoires. Ces indications étaient nécessaires et précieuses car elles amenaient des informations concernant le processus de séparation et d'individuation de l'adolescent, et sur la façon dont il le vivait et l'expliquait. De plus, cette partie devait introduire notre objet d'étude de la pare-excitation en questionnant le sujet sur les activités qu'il  jugeait l'apaiser ou bien sur le moyen de se procurer de l'apaisement.

La deuxième partie consistait à recueillir les données qui interrogent le passage à l'acte du sujet, mais surtout sa perception de celui-ci. Elle devait permettre de saisir la place de l'acte dans la dynamique psychique du sujet, mais aussi de placer l'agir dans le contexte institutionnel afin d'en comprendre l'effet. Cette partie devait donc aider à appréhender le fonctionnement psychique de l'individu par rapport à son acte, tels que son traitement de l'excitation, son niveau d'élaboration psychique et son engagement dans la relation d'objet (l'institution). Ces données devaient permettre, à travers le regard du sujet, de comprendre l'échec de la fonction de pare-excitation de l'institution dans le contexte du passage à l'acte.

La troisième partie portait sur l'investigation du  rapport du sujet à l'institution. Elle devait aider à nous donner des éléments de compréhension concernant son rapport au cadre de l'institution, avec ses règles et ses lois, mais également de comprendre son rapport avec l'équipe de professionnels. Il s'agissait de soulever, à travers sa perception, comment le sujet se situe dans cette dynamique institutionnelle et interprète sa place au sein de l'institution. De plus, cette partie visait à mettre en évidence si le principe d'intervention institutionnel, avec la fonction contenante de son cadre thérapeutique, éducatif et pédagogique, participait d'une figuration psychique.   Enfin, il s'agissait d'identifier ce qui faisait pare-excitation dans l'Institution chez le sujet adolescent au comportement violent.  

\subsection{Le Test de Rorschach}

Selon Chagnon\index{Chagnon} et Cohen de Lara\index{Cohen de Lara} (2012), les tests projectifs permettraient de repérer chez le sujet son mode de relation aux objets, sa problématique des limites et de l'externalisation du conflit, (dont l'impact de l'environnement), ses modalités défensives (narcissiques et archaïques) et ses modalités de traitement des motions pulsionnelles. Toutes ces modalités nous offrent des données importantes qui nous permettront de compléter nos objectifs de recherche et nous amèneront des éléments quant au fonctionnement psychique du sujet.  Pour les auteurs précités qui ont étudié le fonctionnement psychique des enfants instables, le bilan projectif permettrait de mesurer le surinvestissement relationnel d'un sujet à travers un appel à l'autre, ses capacités de pare-excitation, de contenance et d'étayage. Ceci pouvant confirmer la difficulté à fonctionner seul, ce que nous pourrons mettre en lien, concernant notre mémoire, avec la dépendance du sujet à son environnement, à sa recherche de pare-excitation dans l'institution.

Le  test de Rorschach permet de repérer les aspects du fonctionnement psychique, par un jeu de perception et de projection. L'hypothèse de Rorschach est que la personnalité du sujet se dévoile par la façon dont celui-ci résout les problèmes perceptifs, cognitifs et affectifs qu'il rencontre à chaque planche. Ce test possède deux types de planches : les planches compactes et les planches bilatérales. Les planches compactes, dites unitaires,  renvoient à la problématique identitaire, à l'image du corps et à l'intégrité du Moi. Ces aspects seraient intéressants à approfondir dans le cadre de ce mémoire qui étudie la population adolescente. Effectivement, comme nous l'avons démontré précédemment, le phénomène de l'adolescence soulève d'importants changements, comme la modification du rapport au corps, un remaniement narcissique profond et un réaménagement identitaire. Ces planches pourraient donc étayer davantage les données sur le vécu psychique de l'adolescent face à cette transformation et sur les moyens défensifs opérés par celui-ci pour préserver l'intégrité de son Moi. Les planches bilatérales vont plutôt renvoyer à des problématiques de relations. En ce qui nous concerne, nous avons souligné l'importance de la relation au premier objet  et à son rôle dans l'advenue des fonctions symboliques du sujet.

Zabci\index{Zabci}, Ikiz\index{Ikiz}, Kay\index{Kay}aalp et Baudin\index{Baudin} (2005), dans leur recherche sur l'organisation psychique de l'enfant instable à travers le test de Rorschach, remarquent un fonctionnement cognitif et affectif particulier. Concernant le  fonctionnement cognitif de ce public, ces auteurs relèvent un refus des planches, qui traduirait un refoulement et une anxiété intense par un évitement de la verbalisation mais aussi de la pensée. Ces résultats concordent avec l'idée que l'enfant instable s'exprime par l'agir, et qu'il est dans l'incapacité à faire appel à une mentalisation et verbalisation, et donc une difficulté à symboliser les représentations et les affects. Cette difficulté à élaborer une représentation stable  témoigne selon ces auteurs d'une difficulté d'identification et d'une recherche d'un cadre contenant face à des risques d'intrusion désorganisante. Or, cette recherche d'un cadre contenant mise en lumière par les réponses instables et incertaines des enfants, nous intéresse dans ce mémoire afin de  pouvoir justifier la fonction de pare-excitation de l'institution chez l'adolescent au comportement violent.

Cohen de Lara\index{Cohen de Lara} (2000), dans sa recherche menée sur une population d'enfants présentant des troubles du comportement,  a observé chez eux la rareté des productions symboliques, la fréquence des représentations vagues, les défauts d'intériorisation des objets, la fragilité des limites entre l'interne et l'externe, la sensibilité aux modifications de l'environnement et l'évitement du conflit. Concernant le fonctionnement affectif, Zabci\index{Zabci} et al. (2005),  constatent une augmentation considérable des réponses couleur qui marque une extrême réactivité de l'enfant instable et sa dépendance à l'environnement. Toutes ces données sont nécessaires car elles justifient le choix du test projectif qui peut mettre en lumière la dépendance de l'environnement dans le cas des adolescents violents.

De plus, ce test nous renseignera  sur les rapports de l'adolescent avec ses relations précoces, éléments primordiaux pour répondre à notre hypothèse. Selon Chabert\index{Chabert} (1997), « les réponses C' ont à voir avec les relations précoces. » L'apparition de ces réponses, surtout aux planches maternelles (I, VII, IX) met l'accent sur l'aspect d'insécurité de ces modalités relationnelles. De plus, selon lui, la planche VII est considérée comme la planche maternelle par excellence et s'inscrit toujours dans un jeu dialectique des relations au premier objet et la planche IX renvoie à l'imago maternelle archaïque. Zabci\index{Zabci} et al. (2005),  soulignent à travers les résultats du test de Rorschach, la possibilité d'un sentiment de vide, qui doit être pris en considération avec le nombre élevé de réponses C', qui révèlent une angoisse de perte d'objet. Cette centration sur le vide s'inscrit dans le contexte d'une faille, d'un manque qui porte l'accent sur l'incomplétude. Selon ces auteurs, il est possible que l'existence du blanc sur les planches soit ressentie comme manque fondamental par ces enfants, carence dans les premières relations avec la mère. 

Pour eux, l'extrême réactivité aux stimulations externes (résonnance immédiate non élaborée, proches des décharges pulsionnelles, telles les réponses « feu » et « sang ») démontre une fragilité de l'enveloppe psychique qui s'efface quand l'excitation monte.

Selon Zabci\index{Zabci} et al. (2005) : « Toutes ces images (nombre élevé de réponse C') traduisent l'importance de la dépendance et la contrainte inéluctable de prendre appui sur la recherche de support et d'étayage. » Les résultats au test du Rorschach viendront ainsi apporter des éléments de réponse sur l'éventuel besoin de contenance ressenti par l'adolescent. Ces auteurs soulignent l'idée qu'à travers les résultats qu'ils ont obtenus au test de Rorschach, ils ont pu déceler une carence dans l'intériorisation et une grande fragilité sur le plan des identifications qui se traduisent par la multiplication des réponses possibles, l'alternance des représentations et la fréquence des représentations vagues, mal constituées, et indifférenciées quant au sexe. « La recherche d'un cadre contenant, solide et structurant, face à des risques d'intrusion destructrice, menaçant l'intégrité corporelle, est patente. Ainsi, il existe une importante fragilité des limites entre l'interne et l'externe et une extrême sensibilité aux modifications de l'environnement. 

Nous utiliserons les principes de cotation du Rorschach de l'enfant et de l'adolescent de Blomart\index{Blomart} (1998) ainsi que les principes d'interprétation projectives de l'enfant et  de l'adolescent de Emmanuelli\index{Emmanuelli} et Azoulay\index{Azoulay} (2008) ainsi que  de l'approche clinique et projective de Chagnon\index{Chagnon} et Cohen de Lara\index{Cohen de Lara} (2012).

\chapter{Résultats}

\section{Cas clinique de Julien}

\subsection{Passation de l'entretien}

Au premier entretien, Julien semblait vouloir comprendre notre rapport à l'institution, et semblait également attaché à la confidentialité de l'entretien. A l'issue de cet entretien, Julien a accepté de réaliser l'entretien de recherche.

Julien est arrivé par lui-même et à l'heure du rendez-vous et nous a même attendu dans la salle d'attente. Au  début de l'entretien, il semble tendu et son regard est fuyant. Il semble toutefois se décontracter lorsqu'il évoque les activités qu'il aime pratiquer. Julien paraît très investi lors de notre passation, et déploie de véritables efforts d'élaboration pour nous expliquer ce qu'il a ressenti lors de son passage à l'acte. Ses explications ont soulevé chez lui des signes de nervosité assez visibles, et malgré nos interventions qui précisaient qu'il pouvait arrêter de répondre à nos questions à tout moment, il a tenu quand même à poursuivre l'entretien : « Non c'est bon je veux continuer ». Julien n'a pas hésité à nous demander de lui expliquer certaines questions auxquels il ne parvenait pas à comprendre le sens. 

\subsection{Analyse de l'entretien}

\subsubsection{Analyse du parcours de vie du sujet adolescent}

Julien est âgé de 15 ans. Il est issu d'une fratrie de trois frères, dont un plus âgé et un plus jeune. Il habite chez ses parents et son petit-frère, et déclare voir fréquemment son grand-frère qui réside dans la même ville. Il se dit proche de lui et l'accompagne même pendant ses activités : « Je m'entends très bien avec lui, je l'accompagne souvent à ses entrainements de boxe ».   Julien et sa famille sont  originaires d'un pays en guerre. Sa famille a fuit le pays quand il avait 3 ans. Il garde toutefois quelques souvenirs : « je me souviens juste de ma rue et de ma maison » dont un particulièrement: «  Je me rappelle qu'une fois on s'était tous cachés dans le jardin et que des hommes avec des armes rentraient dans notre maison ». Nous pouvons mettre en évidence un sentiment d'insécurité prégnant pendant cette période : « mes parents ils ne voulaient pas que je sorte, car ils ne voulaient pas que je meure ». Mais il semble parvenir à retrouver de la sécurité en France : « je me sens mieux en France, ici c'est en état, là-bas, tout est détruit. », « et puis les jeunes de mon âge, ils ont des armes. À 9-10 ans, t'as une arme et c'est normal, ici ils en ont pas, c'est pas que j'ai peur, c'est que je n'en veux pas ». Ils sont passés par de nombreux pays avant de se stabiliser dans leur ville actuelle, depuis 4 ans. Cependant, il maintient toujours un lien avec sa famille restante dans le pays, qu'il voit régulièrement lors de séjours annuels: « J'y vais quand même là-bas tous les étés, et c'est cool je vois tous mes cousins et tout ». Julien paraît être soudé à sa famille et partage même son temps libre auprès d'elle parmi ses activités d'adolescent : « Je passe beaucoup de temps avec ma famille, mais je sors quand même avec mes potes pour me promener, ou au bar pour regarder le foot, ils viennent chez moi, je vais chez eux. » En dehors de l'ITEP, Julien décrit une relation apaisante avec ses pairs : « je leur raconte comment s'est passé ma journée, ma semaine ici, je me sens bien ».

Julien rentre chez lui tous les soirs et les week-ends. Il décrit de bonnes relations avec sa famille dans laquelle il semble se sentir sécurisé : « je m'entends bien avec eux, ils savent comment je suis, mon caractère et tout. » Son père est maçon tandis que sa mère est femme au foyer. Il a également un oncle qui habite dans la même ville que Julien.

Concernant sa scolarité, Julien a intégré l'ITEP après de nombreux incidents à son collège dans lequel il a été accusé d'agressions verbales et physiques envers ses camarades et professeurs : « je les tapais en sortant, je ne les aimais pas trop, je ne sais pas pourquoi. »,  « Je disais « je vais te crever » aux profs quand ils m'engueulaient ou quand j'avais une mauvaise note ». Julien semble relater un sentiment d'inconfort à travers ses souvenirs : « je ne sais pas pourquoi j'aimais pas ma classe, je me sentais mal, en plus à chaque fois ils allaient se plaindre au directeur », « En plus les élèves ils avaient peur de moi ». Julien a acquis les conditions de son admissions au sein de l'ITEP : « c'est après avoir fait ces conneries que le directeur m'a proposé Grèzes ». Julien poursuit alors son parcours scolaire au sein de l'institution mais avoue éprouver du désintérêt pour  la classe : « j'aime pas trop les cours de toute façon je fous rien je préfère être en atelier », « Cette année c'est mieux car j'ai pas beaucoup de classe, j'ai surtout atelier ». Julien est en atelier mécanique, et paraît très investi dans son domaine professionnel : « J'ai fait des stages mais ça ne durait qu'une ou deux semaines moi j'aimerai que ça dure plus longtemps, ça s'appelle être en apprentissage. » Il arrive même à se projeter dans ce domaine : « je veux faire ça, ça me plait la mécanique, l'année dernière je faisais maçonnerie mais j'aimais moins. »

Nous pouvons saisir dans le discours de Julien un délaissement des identifications antérieures constituées par les figures masculines de sa famille: « Avant je faisais de la boxe comme mon grand-frère mais maintenant je fais du rugby », « je veux faire mécanicien, ça me plait la mécanique, l'année dernière je faisais maçonnerie mais j'aimais moins. ». Ce « mouvement de rupture » va marquer le processus d'identification de Julien, qui va faire appel à des expériences qui lui permettront d'investir de nouveaux supports identificatoires : - En parlant d'un autre adolescent - « C'est celui que vous avez vu dans le couloir, avec qui je traine tout le temps, on fait tout ensemble »,  «On faisait que des conneries, je le suivais tout le temps, je n'écoutais plus rien donc on a été dispersé ». L'identité de Julien est également                                                                                                   marquée par sa culture : « je ne supporte pas qu'on me mette du porc dans l'assiette, car je suis musulman et les musulmans ça ne mange pas de porc. » Ce sentiment d'appartenance à un groupe défini, vient alors renforcer sa construction identitaire.

En dehors de l'ITEP, Julien pratique du rugby, activité qui semble apaiser ses débordements pulsionnels: « C'est la première année que j'attaque le rugby. Le rugby ça me défoule, je préfère quand c'est violent. Le foot c'est trop calme. ». À travers cette activité, Julien semble investir son corps comme un instrument de pare-excitation et semble trouver satisfaction à faire rencontrer ses pulsions avec une limite contenante : « Quand je fais quelque chose j'ai pas envie que ça soit calme, je fonce, j'ai envie de foncer dans quelque chose, et au rugby faut foncer dans le tas. Et c'est ça que j'aime bien. » 

\subsubsection{Analyse de son rapport à l'acte}

Julien affirme se mettre en colère : «  ça m'arrive de temps en temps, seulement quand il y a une raison ». Il paraît associer la colère avec un acte violent, impulsif et destructeur qui semble lui échapper : « Je ne me rappelle pas quand je me mets en colère, je vois pas trop ce que je fais, je ne me rappelle pas de tout, tout ce qui est à côté de moi, je le casse. » Julien définit la colère comme une force qui semble s'imposer à lui : « J'arrive pas à me calmer tout seul dans ces moments-là, je le sais, mais ça monte tout seul » L'état de Julien dans ses moments de colère nous fait penser au débordement pulsionnel propre de l'agressivité libre, qui cherche une sorte de contenant : « j'arrive pas à me calmer tout seul ». 

Il déclare souvent se mettre en colère en dehors de l'institution : « En dehors d'ici, ça m'arrive souvent, mais il y avait des raisons. Mais avec ma famille, ça ne m'arrive jamais, ils savent comment je suis, avec mon caractère. » Cependant, Julien ne se rappelle pas d'exemples précis où il a pu se mettre en colère en dehors de l'institution. Il affirme la régularité de ses colères, mais ne parvient pas à se saisir d'exemples concrets.

Au sein de l'institution, Julien affirme se mettre très rarement en colère mais il parvient à citer spontanément un exemple précis : « Ici, ça m'est arrivé juste une fois, quand ils m'ont mis du porc dans l'assiette, car je suis musulman et les musulmans ça ne mange pas de porc. Mais ils n'ont pas fait exprès. Mais j'ai cassé la vaisselle, la table et les fenêtres. Je pouvais pas m'arrêter. Je le savais qu'ils n'avaient pas fait exprès mais c'est monté quand même. À Grèzes, c'est la seule fois où j'ai été violent. » Julien semble relater une effraction de l'objet dans son narcissisme, qu'il tente de reconstruire en s'identifiant à une appartenance d'un groupe défini : « car je suis musulman et les musulmans ça ne mange pas de porc ». Cette situation nous fait penser au précepte de Balier\index{Balier}, qui stipule que le passage à l'acte est un aménagement défensif contre l'effraction de l'objet dans le narcissisme du sujet. 

L'identification de Julien qui s'établit par un sentiment d'appartenance à un groupe culturel défini semble participer à l'élaboration de sa propre identité. Son passage à l'acte peut alors être interprété comme une tentative de séparation et d'individuation, car il s'est peut-être senti assigné en tant qu'objet de l'institution. Son sentiment de sujet étayé par une particularité identitaire définie s'est peut-être retrouvé anéanti, puisque celle-ci menacée. L'intrusion de l'objet dans son narcissisme fragilisé a provoqué ce débordement pulsionnel qui n'a pas pu être élaboré et mentalisé.  Ceci  a engendré ce passage à l'acte, qui pourrait constituer  une tentative  de symboliser cette séparation et individuation. Pourtant, Julien a précisé avoir intégré le fait que cet acte était involontaire de la part des professionnels : « je savais pourtant qu'ils n'avaient pas fait exprès. » indication qui semble aller dans le sens d'un débordement pulsionnel.  

Julien décrit, dans un premier temps, un phénomène qui semble s'imposer malgré lui et qui semble même agir en dehors de lui : « c'est monté quand même ». Cette agressivité libre, va chercher une sorte de contenant : « J'ai cassé la vaisselle, la table et les fenêtres ». Julien parvient à citer des exemples sur des moyens qui pourraient le calmer à ce moment-là : « Par exemple, s'il y a quelqu'un à côté de moi à ce moment-là, et qu'il ne me dit pas son prénom, je ne vais pas savoir, alors je vais le taper. » Nous avons tenté de comprendre en quoi s'agissait ce savoir : « Savoir quoi, pourrais-tu me l'expliquer ? » Mais Julien n'a pas su en dire davantage malgré beaucoup d'efforts : «  Je ne sais pas comment l'expliquer, pourquoi je ne sais pas, c'est trop bizarre… » Julien poursuit son discours et explique qu'il n'y a qu'une solution pour le calmer : « Pour me calmer, il doit me mettre par terre, il doit me dire son prénom, après je me calme et ça revient en moi. » Julien n'a pas su expliquer, ce « qui revenait en lui », mais semble s'appuyer sur deux moyens de calmer ce débordement pulsionnel qui semble s'imposer à lui. Tout d'abord, par un contact physique contenant, une force qui va le contenir, qui va empêcher cette agressivité libre qui aura finalement trouvé une sorte de contenant par la force physique de l'éducateur qui peut se traduire par une sorte de « pare-excitation physique ». Et enfin, la deuxième solution est de « dire son prénom ». Cette solution paraît être pour Julien un moyen de « pare-excitation verbale » pour lui. Peut-être que se présenter en disant son prénom, lui rappelle la fonction occupée de la personne au sein de l'institution.

Avant le passage à l'acte, Julien semble être envahi par un débordement pulsionnel qui se manifeste physiquement : « c'est monté d'un coup et j'ai commencé à trembler » qu'il ne parvient pas à maitriser. Il évoque un anéantissement de l'organisation psychique qui s'ensuit d'une absence de mentalisation significative : « je ne sais pas ce que je fais à ce moment-là, je ne me rappelle de rien, je ressens rien, ni la douleur quand je m'énerve ». Après le passage à l'acte, après avoir été contenu physiquement au sol par les éducateurs, Julien ne se souvient pas de ce qu'il a pu faire ou détruire : « Quand je suis en colère, je ne sais pas ce que je fais, ce sont les éducateurs qui m'ont expliqué ». C'est dans l'après-coup que Julien réalise son acte par les propos des éducateurs : « Dans ma tête je me dis que je suis allé un peu loin, mais ça me rassure quand ils me disent ce que j'ai fait. »

\subsubsection{Analyse de son rapport à l'institution}

Julien est au sein de l'institution depuis un an. Il est en atelier mécanique et est  externe.

Tout d'abord, nous pouvons souligner que cet entretien a permis de repérer que le rapport de Julien à l'Institution se réduit à l'équipe éducative. Lors de l'entretien, Julien ne paraît pas comprendre lorsque l'on évoque l'Institution. À la question : «  Quelles mesures ont-été prises par l'institution après cet incident ? », Julien déclare ne pas comprendre la question, une reformulation a donc du être opéré : « Qu'est ce qui s'est passé après cet incident ? ». C'est à ce moment-là qu'il a pu répondre : « ils m'ont collé par terre, ils m'ont dit d'arrêter, ils m'ont calmé, après ils ont appelé mes parents. » Ce qui semble avoir fait pare-excitation chez Julien, dans cette situation, se réduit à sa relation avec ses éducateurs. Ce périmètre semble se réduire à l'équipe éducative, sans faire référence à l'institution avec son équipe pluridisciplinaire et de direction. Cette équipe éducative semble constituer un cadre sécurisant pour Julien : « Avec eux, je me sens bien, je suis bien. »

Les lois et les règles semblent avoir été intégrées et respectées par Julien : « le respect, je dois pas voler, pas me battre, tout ça…il faut arriver à l'heure », « Quand tu fais une connerie ici t'es collé » et reconnaît par ailleurs la légitimité de la loi institutionnelle : « après m'être énervé j'ai été puni, même si ça me fait chier, c'est normal, je suis allé trop loin ».

Julien décrit une relation sécurisante avec son éducateur référent : « Simon, c'est mon éducateur perso à moi. C'est lui que je vais voir dès que j'ai un problème à Grèzes. Il m'aide pour le rugby, j'ai vraiment de bonnes relations avec lui, ça fait deux ans que je le connais, je me confie à lui aussi. C'est le seul avec qui je m'entends comme ça. »

Plusieurs projets ont été mis en place par l'institution pour Julien, projets qui se révèlent pour lui être un déclic professionnel : « J'ai fait des stages mais ça ne durait qu'une ou deux semaine, moi j'aimerais que ça dure plus longtemps, ça s'appelle être en apprentissage. » L'existence de ce projet semble apporter chez Julien, un cadre sécurisant bénéfique, valorisant,  comme nous pouvons le relater quand ils énonce ses moments préférés au sein de l'institution : « Quand je suis en mécanique, c'est ce que je préfère, car ça me plait, j'y arrive, et c'est ce que je veux faire plus tard et puis j'aime bien le prof » tandis que les moments où Julien se sent le moins bien, peut-être mis en rapprochement par un manque d'intérêt pour lui  : « Pendant les cours, ça me saoule, je fous rien. »

Julien connaît son emploi du temps, et arrive à repérer ses horaires concernant ses activités mécaniques, ses activités sportives et ses heures de classes. « Le matin on a école, puis après l'aprem j'ai atelier et le soir ça finit à 5h, du coup je rentre chez moi et les internes ils rentrent dans leurs groupes. »

\subsection{Analyse Clinique}

\subsubsection{Problématique objectale et narcissique du sujet}

Nous pouvons dire que Julien est marqué par un déracinement culturel, c'est pourquoi il est important de prendre en compte la dimension interculturelle de son cas.  À travers ses propos, nous pouvons constater que malgré l'investissement de ses parents, Julien a grandi dans un environnement insécurisant où il a été confronté à l'angoisse de mort dès son plus jeune âge : « mes parents ils ne voulaient pas que je sorte car ils ne voulaient pas que je meure ». Cette angoisse de mort est activée par la propre angoisse de mort de ses parents. L'objet parental, dans le contexte d'un environnement défaillant et anxiogène, n'a peut-être pas pu assurer à Julien un climat fiable et sécurisant. De plus, c'est à l'âge de trois ans que Julien a quitté son pays afin de parcourir plusieurs pays en tant que réfugié politique. Cet environnement marqué précocement par la guerre et un long déracinement a peut-être impacté l'établissement de son narcissisme primaire. L'investissement de Julien pour l'objet parental  n'a peut-être ainsi pas été perçu comme suffisamment sécurisant. Nous pouvons donc émettre l'idée que l'élaboration des assises narcissiques de Julien a été perturbée par son contexte environnemental lors de son enfance. Ce sentiment d'insécurité va réactualiser au collège un mode d'aménagement défensif centré sur la violence. Julien va faire ainsi appel à des comportements hétéro-agressifs au sein de cet établissement.  

Concernant la construction de son identité, Julien s'est étayé sur les personnes significatives de son entourage dont son père et son frère, ce qui lui a permis d'établir une bonne identification à ces supports d'identifications antérieures. Les planches IV et VI du test de Rorschach sont associées à des figures masculines, paternelles et/ ou viriles. Elles réactivent le vécu par rapport au masculin et à la sexualité virile. Cette identification virile paraît être de bonne qualité selon les réponses de Julien, malgré une certaine résistance au départ, à la planche VI, que nous pouvons interpréter comme un signe défensif. La planche IV est d'ailleurs une des planches préférées de Julien, car « j'y vois pleins de choses ». Cependant, les planches II, VII et IX ont été notées par un refus significatif, concernant les deux premières, marquées par des exclamations. Il a pu toutefois fournir des éléments de réponses, fragiles, mais qui n'ont pas été rappelé lors de l'enquête. Ces planches réactivent le vécu face au maternel féminin en terme de perception et de réaction, et renvoient à l'acceptation de la sexualité féminine et à la reconnaissance de la castration d'une imago maternelle phallique. Nous pouvons noter que chez Julien, ces processus font l'objet d'une réaction que nous pouvons interpréter comme produite par une défaillance de l'intégration de l'imago maternelle ou par une imago maternelle insuffisamment structurante. Toutefois, la planche IX, par ses couleurs, fait appel à la régression du sujet et réveille son vécu face à une imago maternelle primaire, et Julien par ses réponses semble avoir pu fournir une réponse de bonne qualité grâce à l'attraction des couleurs de la planche.

Les planches III et V font appel à la capacité du sujet à s'identifier à une représentation humaine et animale de bonne qualité ; ce dont a fait preuve Julien. Cependant, nous pouvons noter que lors de la réponse à la planche III, les références de Julien à une radio d'un squelette, avec ses os, que nous pouvons interpréter comme une effraction des limites et de l'impact de l'environnement, livrent ainsi certains signes d'une fragilité des limites entre soi et le monde extérieur. Nous pouvons penser que l'enveloppe psychique et sa représentation interne n'assure pas chez Julien une propriété contenante et protectrice face aux excitations internes et externes. C'est d'ailleurs par son activité sportive de prédilection, le rugby, que Julien semble investir cette frontière poreuse, entre le dedans et le dehors, entre son fonctionnement psychique et la limite de l'environnement. Son corps semble être investi comme un support possible à la quête de cette frontière : « j'ai envie de foncer dans quelque chose, et au rugby faut foncer dans le tas. Et c'est ça que j'aime bien ». 

Le pôle sensoriel de Julien est faible (C > K), ce qui renvoie à une porosité des limites (Emmanuelli\index{Emmanuelli} et Azoulay\index{Azoulay}, 2008). La réponse à la planche IV renforce cette idée : « on dirait une grande lave qui sort des montagnes ». Ceci met en jeu un mouvement d'oscillation entre « contenance » et « effraction » qui renvoie à une forme de mise en échec de l'établissement des frontières interne/externe et des frontières du Moi. La fragilité des frontières qui séparent le Moi des motions pulsionnelles ainsi que du dedans et du dehors laissent présager une faiblesse des défenses du Moi de Julien et de son système de pare-excitation. Son protocole est marqué par des réactions comportementales, par des exclamations ou des sidérations face aux changements de stimulus, et parfois même des refus.  Julien   peut de ce fait tenter de se protéger face à un risque d'excitation qui menacerait de l'envahir en mobilisant massivement des ressources d'opposition. Toutes ces caractéristiques soulignent une fragilité des limites et un impact perceptible de l'environnement (Chagnon\index{Chagnon} et Cohen de Lara\index{Cohen de Lara}, 2012).

À travers son discours, Julien semble se défaire de ses supports identificatoires antérieurs constitués par les objets masculins de sa famille, afin de se constituer sa propre identité par l'investissement de supports identificatoires externes. Ce processus de séparation et d'individuation, réactualisé lors de l'adolescence, est un processus permettant la construction de l'identité. Dans le cas de Julien, selon ses propos, ses supports identificatoires externes concernent un de ses pairs au sein de l'institution et son activité de mécanique, qu'il investit comme sa future profession. Son appartenance à un groupe défini : « car je suis musulman et les musulmans ça ne mange pas de porc » constitue une des facettes de son identité. Cependant, cette identité qu'il tente de construire par ses supports identificatoires est fragile, car Julien est dans la période de l'adolescence où le Moi se retrouve confronté aux conflits dynamiques propres de cette période. De plus, son narcissisme est atteint par le stigmate de l'environnement insécurisant et instable dans lequel il a grandi. Cet environnement n'a peut-être permis l'établissement des bonnes assises narcissiques dont Julien aurait eu besoin pour construire une bonne intériorisation de l'objet primaire. Nous pouvons noter une défaillance de contenance psychique de Julien, ce qui va se définir par des modalités de traitement des motions pulsionnelles particulières.

\subsubsection{Modalités de traitement des motions pulsionnelles}

D'après les propos de Julien, sa réaction au conflit s'exprime par un évitement de la pensée et donc par un refus d'élaborer. Nous pouvons nous rendre compte que lorsqu'il est confronté à un objet suscitant une tension, il n'est pas possible pour lui de traiter psychiquement cette charge d'excitation. Cette tension psychique qui se répercute sur le plan physique (tremblements), au détriment d'une mentalisation, va chercher un contenant afin d'éviter la désorganisation qui suivrait un anéantissement psychique. Ceci souligne la difficulté de Julien à se saisir de sa propre pare-excitation. Il va chercher à l‘extérieur de lui-même ce qui pourra contenir ses pulsions. Dans le cas du passage à l'acte cité par Julien, ce sont les éducateurs, par leur  force  physique, qui vont contenir la charge d'excitation de Julien.

D'après les résultats de la passation du Rorschach, nous pouvons souligner que lorsque Julien s'est trouvé confronté à la planche II, planche bilatérale qui réveille l'angoisse du sujet face aux émergences libidinales et agressives, celle-ci n'a pas pu être canalisée et a paru entraver le processus de réponse part une forte inhibition, qui s'est soldée par un refus. Nous pouvons interpréter ce refus par une difficulté à mentaliser cette émergence libidinale et agressive, Julien s'est alors confronté à une tension qui a amputé toute élaboration.

Dans le cadre de son passage à l'acte, Julien n'a pas su élaborer le conflit psychique qu'a suscité la vue de la présence de porc dans son assiette. L'effraction de l'objet a peut-être conduit Julien à se sentir dissous dans une fusion avec l'objet institutionnel. Cette situation a suscité chez lui une tension qui a totalement réduit à néant toute forme d'élaboration. Le caractère anti-narcissique de l'objet, ici l'institution, contre lequel le sujet se défend par le passage à l'acte, est révélateur de l'incomplétude du sujet et de sa dépendance à l'égard de l'objet (Racamier, 1992). Cette dépendance étant que l'objet, l'institution, représente une figuration psychique pour Julien.

\subsubsection{Les effets de la figuration psychique de l'institution sur Julien}

Comme nous l'avons vu, le fonctionnement psychique de Julien semble être marqué par une dépendance à l'environnement. Julien souffre visiblement d'une défaillance de contenant symbolique du fait de son manque d'élaboration face à une tension psychique. Il va ainsi dépendre de l'environnement donc dans ce cas, l'institution, afin de traiter ce que l'appareil psychique ne peut pas traiter (Jeammet\index{Jeammet}, 1985). 

C'est ce que Julien a mis en évidence lors de ses indications concernant son passage à l'acte. La fonction de pare-excitation qui est normalement dévolue au fonctionnement mental, va être attribuée pour Julien, à l'institution, notamment, à ses membres qui la représentent dont particulièrement l'équipe éducative. Pour lui, lorsqu'il se confronte à une tension, celle-ci paraît insurmontable, il est dans l'incapacité à gérer seul ce conflit. Dans ces situations, seul le contact physique des éducateurs qui le plaquent au sol parvient à le calmer. La personne qui se situe auprès de lui pendant son passage à l'acte, pourra l'apaiser, d'après ses propos, en se présentant par son prénom. Lorsque Julien est en situation de crise, ce qui fait pare-excitation pour lui, est le contact physique des éducateurs. Par ailleurs, la présentation des membres qui se situent près de lui, comme pour préciser leur fonction et leur place, participent peut-être pour Julien à une possibilité de mentalisation et d'intégration des intervenant participant au cadre institutionnel, qui constitue une figuration psychique pare-excitatrice qui semble fonctionner chez Julien en temps normal.

Julien semble avoir intégré les lois et les règles, ainsi que le cadre. Le cadre, permettant un système de temps, de lieu et d'action par l'appareillage psychique des intervenants comme le précise Pinel\index{Pinel} (1996), est un principe méthodologique d'intervention institutionnelle. Julien arrive à représenter ses journées dans le temps avec ses lieux et ses actions. L'intégration de ces lois, du règlement, du temps, de l'horloge et de l'espace géographique, participe à une action contenante, et joue sur les limites du Moi, c'est ce que Bergeret\index{Bergeret} (1988) a nommé « psychothérapie institutionnelle ». Cette action contenante fonctionne visiblement chez Julien. C'est ce que nous avons repéré, notamment lorsqu'il ne parvient pas à se souvenir de ses crises de colères en dehors de l'institution. Il évoque pourtant le fait de se mettre souvent en colère en dehors de l'institution mais n'arrive pas à se souvenir d'exemples. Cependant, c'est spontanément qu'il fait référence à un passage à l'acte qu'il a commis une fois dans l'institution. Julien a pu même se souvenir de son lieu précis et de sa date. C'est ainsi que nous pouvons mettre en exergue le périmètre de contenance psychique permis pour Julien par l'institution. C'est grâce à l'intégration de ses lieux, de ses horaires, de ses différentes activités, de ses lois, de son règlement ainsi que de la fonction des différents intervenants, que Julien a pu se rappeler précisément de son passage à l'acte.

De plus, Julien est investi par son activité de  mécanique. Il aimerait d'ailleurs en faire sa profession. Plusieurs projets ont été mis en place pour lui, il a fait plusieurs stages chez des garagistes, et attend d'être en apprentissage afin de passer plus de temps sur les lieux de stage. Julien semble à la fois investi mais surtout contenu par ce projet professionnel. Kaës\index{Kaës} (1996) souligne que l'instauration de l'objet et des objectifs permettra de comprendre ce qui de la réalité psychique est mobilisé, étayé et immobilisé par l'institution. D'après Marty\index{Marty} (2007), l'institution ne va se constituer en pare-excitation que si elle est capable de contenir l'excitation pulsionnelle, de lui donner forme et limite, puis de lui donner sens. L'activité  de mécanique de Julien semble contenir son excitation pulsionnelle, en lui donnant forme et limite (réalisation concrète sur le plan mécanique) puis de lui donner sens. Ses moments préférés au sein de l'institution se réfèrent à cette activité qu'il déclare être sa future profession. 

Cependant, nous pouvons relever l'échec de la fonction pare-excitatrice de l'institution, du fait du passage à l'acte de Julien. D'après les éléments que nous avons amenés précédemment, c'est justement de cette dépendance à la figuration psychique qu'incarne l'institution que Julien a voulu se défaire pour tenter de s'individualiser en se séparant, à défaut de symbolisation, par le passage à l'acte. 

Pour Julien, l'institution peut-être rapprochée de l'objet-mère. 

Le premier objet permet le processus de séparation et individuation, par l'agressivité libre de l'enfant. Si celle-ci est contenue par l'objet-mère, l'enfant pourra accéder aux représentations symboliques. Grâce à l'Institution qui contiendra sa décharge pulsionnelle et palliera ainsi son propre défaut de fonction symbolique, Julien pourra ainsi symboliser la séparation et l'individuation. 

\section{Cas clinique de Marc}

\subsection{Passation de Marc}

Marc paraît être décontracté au commencement de l'entretien. Il n'a aucune difficulté à nous livrer son passé. À la présentation de notre entretien de recherche et à nos rappels concernant le code de confidentialité : « Tout ce que tu diras ne sera pas livré à l'institution, je suis soumise au secret professionnel » Marc semble se détacher de cette formalité : « oui pas de soucis, je m'en fous, pas de problèmes ». Marc nous donne l'impression de mener l'entretien, par ses longues digressions, ce qui nous a compliqué son déroulement logique, guidé par notre questionnaire. Concernant ses propos sur son rapport à l'acte, Marc semble agité et être gouverné par ses pulsions et son élaboration demeure très faible : « Je sais pas pourquoi, il me saoule, du coup je le frappe. » Ce qui a compromis certaines de nos données concernant le passage à l'acte.

Au moment de l'entretien de recherche, Marc se saisit de post-it jaunes en forme de carrée, et réalise un cadre jaune sur la table.

\subsection{Analyse de l'entretien}

\subsubsection{Analyse du parcours de vie du sujet adolescent}

Marc est âgé de 13 ans, il a un  grand-frère de 17 ans et une petite-sœur de 2 ans : « je m'occupe tout le temps d'elle, elle crie un peu beaucoup mais j'aime bien m'occuper d'elle. » Son grand-frère est cuisinier et Marc semble décrire une relation conflictuelle avec lui : « au début, on ne s'entendait pas du tout, on se donnait des coups de poings. Une fois, il m'a même griffé fort et j'en garde une trace » (il nous montre sa cicatrice sur la joue). C'est à partir de l'évocation de la relation avec son frère qu'il aborde par lui même son contexte familial « je lui en veux car j'avais besoin de lui, c'était dur. Ma mère est partie quand j'avais trois ans. Il y avait des coups de téléphone dans la maison, je ne savais même pas qui c'était. Et vu que j'étais le plus petit, mon père me protégeait, ce qui est normal mais mon frère il a été un peu mis de côté alors il m'en voulait mais maintenant ça va mieux. Il a compris que ce n'était pas de ma faute. » Marc décrit alors de bonnes relations actuelles avec son frère, avec qui il partage ses activités d'adolescent : « Maintenant ça va, on sort en ville, chez des potes, on fait du vélo, on s'amuse ». Marc a vécu chez son père jusqu'à l'âge de 9 ans. Père qu'il décrit protecteur dans un premier temps, « mon père me protégeait » mais rapporte avoir vécu des maltraitances physiques : « Quand j'étais petit, il me battait. J'ai eu des problèmes à cause de lui, c'était une personne qui buvait beaucoup, à des moments, il était ivre. C'était pas facile pour moi. C'est moi qui ai demandé avec mon frère qu'on me place chez ma mère », « Un peu plus tard, je voyais ma mère une fois par mois, et elle voyait bien que ça n'allait pas ».  Depuis, Marc vit chez sa mère, avec son beau-père et leur enfant, petite fille âgé de 2 ans : « J'ai vécu avec mon père et mon frère, mais là ça fait 4 ans que j'habite chez ma mère. » Un placement qui s'est révélé pour lui, difficile : « Au début, ça n'allait pas du tout, je pétais des câbles tout le temps, et puis je ne pouvais pas voir mon beau-père. » Cette relation conflictuelle semble avoir évolué  : «  Mais bon, quand je suis arrivé chez ma mère, je ne savais pas lire et pas écrire. J'étais en CE2 et j'avais déjà redoublé deux fois. J'avais un niveau CP. J'étais en CE2 au lieu de CM2. Alors mon beau-père il m'a appris à lire, à écrire, il me rabâchait tout le temps, il me faisait chier, mais sans lui je ne sais pas comment j'aurai fait. Parce que lui c'était pareil, il ne savait pas lire, ni écrire à mon âge. » L'apport bénéfique de son beau-père fait naître chez Marc les prémices d'un support identificatoire mais qui sera toujours établi sur un mode comparatif avec son propre père : « Mon père il est dyslexique dysorthographique, alors c'est pas lui qui m'aurais appris. Mais pire qu'aggravé. Il ne sait même pas remplir un chèque ». Le support identificatoire que figure le beau-père va même impacter son projet professionnel : « Plus tard je veux rentrer dans l'armée. Ca fait un moment que je veux rentrer. Le père de mon beau-père c'était un ancien militaire. Ils ont tous fait ça, c'est de famille. Et du côté de ma mère, ils ont tous fait militaire aussi. Contrairement à mon père, lui pas du tout. »  Mais Marc semble souffrir de l'absence de son père et également de la situation précaire de celui-ci : « Ca me saoule parce que mon père, je ne le vois plus. Il habite loin, il n'a pas de travail, il a des problèmes financiers, il vit dans un garage. »

Marc est en internat durant la semaine à l'ITEP et le week-end il rentre chez sa mère : « Je suis en internat, je rentre chez ma mère le week-end, et je sors avec des amis. » Internat, dans lequel il ne semble pas avoir crée de bonnes relations avec ses pairs : « bof, je m'y sens pas super bien, il y a que des gamins dans mon groupe, ils font n'importe quoi. Mes amis il sont en dehors d'ici. », « Ici, il y a quand même Jérémy, je le considérais comme mon meilleur ami, mais ça s'est mal passé, on s'est battu, car il m'avait mal parlé, mais maintenant ça va, on se reparle, quand on est sur l'atelier il est gentil avec moi. » 

Marc semble passer régulièrement par le conflit pour introduire une relation positive. Que ce soit avec son frère, son beau-père, son ami Jérémy et finalement son éducateur : « Dans l'institution j'aime bien Fabrice, avec lui, dès que je suis énervé, j'arrive à discuter, il m'a toujours soutenu. Mais au début, je m'entendais pas du tout avec lui, je le détestais. » Ce mode relationnel ambivalent, que nous pouvons rapprocher à ses relations primaires objectales, une mère totalement absente puis trop présente : « elle est toujours sur mon dos », un père protecteur mais violent. 

Marc était scolarisé dans un établissement extérieur à l'Institution, mais semblait ne pas être au niveau attendu : « J'étais au collège et j'y suis plus. J'étais en 6ième au lieu de la 4ième. Ils m'avaient mis à temps plein mais je n'y arrivais pas alors ils m'ont mis juste le matin. » Mais depuis, Marc a réintégré la scolarité au sein de l'ITEP en raison de comportements violents au sein du collège. 

Le propos de Marc « ils m'avaient mis à temps plein » semble évoquer une confusion au niveau de ses choix, qui semblent selon ses propos appartenir à celle de l'institution. Il précise régulièrement qu'il n'est jamais à sa place : « j'étais en 6ième au lieu de la 4ième », « J'avais un niveau CP. J'étais en CE2 au lieu de CM2 ».

Quand il est chez lui, Marc préfère s'adonner à des activités extérieures au domicile: «  Quand je rentre chez moi le week-end, je suis toujours à l'extérieur. Je traine avec des potes, c'est juste pour pas être chez moi, même quand il pleut je ne veux pas être chez moi. Ma mère me saoule, je dois toujours m'occuper de la petite, t'as voulu une gamine, tu t'en occupes. » Ses propos semblent également interroger la place qu'il occupe au sein de cette dynamique familiale.

Les activités de prédilection de Marc se résument principalement à des activités sportives : « J'adore le sport, j'ai fait deux ans de boxe, j'ai fait du foot, du rugby, 2 ans de hip-hop et là je vais faire de la gym » qui selon lui, lui apportent un effet qu'on peut définir d'apaisant : «  ça me fait du bien, ça me fait souffler, ça me défoule » Nous pouvons émettre l'hypothèse que le rapport au corps de Marc se figure pour lui  à la fois en tant que possibilité de décharge pulsionnelle, et comme attente d'une solution d'apaisement.

Les autres activités de Marc se résument également à des activités physiques qui dépendent de l'environnement : «  Je fais un peu de trottinette, j'ai fait du skate pendant un moment, de la natation pour le plaisir…ça me soulage, ça me détend. » Pour Marc, l'apaisement se réfère systématiquement à une activité sportive ou physique, toujours motrice. 

\subsubsection{Analyse de son rapport à l'acte}

Selon Marc, ses phases de colère se sont atténuées : « Je ne me mets plus trop en colère en ce moment… » Mais il relate de nombreux incidents où il s'est montré violent. Marc semble associer la colère au passage à l'acte et relate par lui même un événement au sein de l'institution, notamment dirigé vers l'équipe éducative : « à des moments ici, j'ai insulté un éducateur, je l'ai insulté, lui et sa collègue, et il m'a engueulé, je l'insulte, il me tient, je lui dit : « lâche-moi » et je lui ai donné un coup de poing. J'ai été collé mais il y a eu tellement pleins d'histoires… » Dans cette situation, nous pouvons interpréter le passage à l'acte hétéro-agressif de Marc, comme une réponse à la contention physique de l'éducateur sur lui. Ici, le passage à l'acte pourrait être interprété comme une tentative de destruction du lien à l'objet en vue de protéger le narcissisme de Marc, comme un mouvement de « désobjectalisation » (Jeammet\index{Jeammet}, 1989). Cette effraction physique de l'objet (Balier\index{Balier}, 1988) 1a précipité Marc dans le passage à l'acte afin de défendre son narcissisme menacé. Marc ne parvient pas à expliquer et à mettre en lumière l'élément déclencheur de son passage à l'acte. Il nous semble qu'il ne parvient pas lui même à extraire un élément qui aurait pu lui permettre de comprendre celui-ci. Lorsqu'il tente d'expliquer ce qui aurait pu provoquer une telle réaction chez lui, ses propos se réfèrent la plupart du temps à : « j'sais pas pourquoi, il m'a saoulé ». Marc arrive à évoquer la souffrance qu'induit les débordements pulsionnels qui semblent s'imposer à lui : « C'est pas un amusement d'être énervé pour rien ».

Marc déclare parfois tenter de changer l'objet de son agressivité lorsqu'il se retrouve dans des situations de colère : « J'ai pété le mur de ma chambre trois fois. À la place de taper les jeunes et les adultes, je tape le mur. » Selon Balier\index{Balier} (1988) la déliaison des pulsions de mort et de vie crée une tension. La tension zéro est recherchée. Celle-ci est atteinte lorsque la décharge de l'agressivité devenue libre trouve une sorte de contenant qui empêche la désorganisation psychique dont le sujet est menacé.  

Marc semble à la fois être violent avec l'équipe éducative mais également avec ses pairs au sein de l'institution : «  Je frappe souvent les jeunes ici, mais après je m'en veux une fois que je suis calmé ». Nous pouvons penser qu'au moment de l'acte, cette décharge pulsionnelle qui semble s'imposer et ne pas s'accorder à sa volonté produit  chez Marc un anéantissement psychique : « Ça sort tout seul » et souligne que lors de son passage à l'acte, celui-ci l'entraîne à faire des choses qu'il ne ferait pas en temps ordinaire : « Il y a beau avoir mon meilleur ami en face, je peux le cogner quand je suis énervé. » C'est seulement après l'accès à la tension zéro que Marc pourra retrouver ses capacités psychiques : « après je m'en veux une fois que je suis calmé ».

En dehors de l'institution, Marc semble avoir trouvé une solution d'apaisement dans le contact direct avec son père : « Dès que je suis énervé, la première personne que j'appelle c'est mon père. » Au sein de l'institution, ce qui le calme selon lui, est une discussion avec un éducateur : « Dans l'institution, il y a Fabrice, dès que je suis énervé, avec lui j'arrive à discuter. Avant il était dans le groupe 6. Maintenant il est dans le 7. Il m'a toujours soutenu. Au début je le détestais. On discute ensemble et ça me calme. » Marc ne parvient pas à se calmer seul, mais va faire appel à l'environnement constitué par ses proches. Nous pouvons relever chez lui une impossibilité à trouver en lui une fonction de pare-excitation , ce qui irait dans le sens d'un défaut de contenant psychique, donc de symbolisation.

Marc revient sur un événement où il s'est mis en colère en dehors de l'institution, à l'intérieur de l'établissement où il a été scolarisé :« Le jour de la réunion, je me suis battu au collège, en fait j'ai voulu séparer un pote avec un mec, et lui il m'a dit : « qu'est ce que tu fais fils de pute » alors là, j'ai frappé, frappé, frappé, frappé…Je supporte pas à ce qu'on insulte ma famille. Alors maintenant je fais l'école à l'ITEP. » La précision du jour de réunion peut nous interpeller dans le sens où ce jour, il était question de débattre de la place de Marc au sein du collège. Mais nous reviendrons sur cet aspect dans la partie concernant le fonctionnement psychique de Marc.

\subsubsection{Analyse de son rapport à l'institution}

Marc a intégré l'institution depuis un an, participe aux activités de l'atelier menuiserie et est accueilli en internat.

D'après ses propos, celui-ci ne se sent pas à l'aise dans son groupe dans lequel  il ne semble pas avoir sa place : « bof, je m'y sens pas super bien, il y a que des gamins dans mon groupe, ils font n'importe quoi, mes amis il sont en dehors d'ici ». Concernant sa relation avec les éducateurs au sein du groupe, Marc semble partagé entre deux d'entre eux qu'ils déclare ne pas supporter : « eux je ne les supporte pas » car : « ils gueulent tout le temps, c'est le bordel avec eux » et apprécier deux autres car: « Avec eux, c'est calme, ils sont cool, et c'est calme ». Marc n'a pas su indiquer d'éléments qui pourraient permettre de comprendre ce que font concrètement les éducateurs pour que ce soit calme. Toutefois, il semble apprécier les intervenants qui peuvent canaliser ses éventuels débordements pulsionnels, ainsi que ceux de ses pairs, ce qui va produire chez lui une sensation qu'on peut qualifier d'apaisante. Les éducateurs ont pour Marc un rôle important de pare-excitation. L'éducateur référent de Marc est selon lui sécurisant et apaisant, donc bénéfique : « il y a qu'avec lui que j'arrive à discuter et à me calmer ».

Au sein de l'atelier menuiserie, Marc déclare ne pas se sentir bien avec le professeur : « Ca va, sauf avec le prof, il me saoule, je ne peux pas le blairer. Dès qu'il parle j'ai envie de le frapper. Alors qu'avec les autres ça passe bien ». Nous pouvons repérer que lorsque nous évoquons une activité, ou bien un groupe, Marc fait référence au professionnel qui l'institue. Bien que la relation avec celui-ci soit entravée par sa problématique objectale, Marc semble avoir intériorisé une part de la fonction contenante de l'institution ; chaque activité et chaque groupe sont reconnus par lui comme tenus  et contrôlés  par un professionnel. Cette modalité appartient au concept de psychothérapie institutionnelle.

Les moments où Marc se sent le mieux au sein de l'institution, sont les activités sportives : « Ma matière préférée c'est le sport, j'adore ça, et puis je suis plutôt doué ». Son idéal du moi lui permet de s'attribuer une valorisation participant d'une gratification narcissique. Marc n'en trouve pas d'autres en dehors du sport. Les moments qu'il préfère le moins sont les temps passés en classe : « les heures de cours, ça me fait chier ». Il est important de souligner que Marc vient tout justement de réintégrer le cadre scolaire de l'institution, car il était scolarisé précédemment dans un établissement en dehors de l'institution.

Concernant les lois, Marc semble les avoir intégré : « Par exemple, t'as pas le droit d'avoir de la « beuh », c'est interdit ici. Dans mon portefeuille j'avais trois têtes de beuh, je suis allé chez les flics, je leur ai dit : « c'est pas à moi », après je leur ai dit d'où ça venait, que c'était moi qui plantait chez moi, mais en fait je l'avais récupéré ici, j'ai pas voulu balancer », « bah t'as le droit de frapper les autres évidemment, déjà ! »

Marc ne semble pas avoir investi et intégré son emploi du temps : « Je sais pas, ça dépend des jours mais je sais plus. Il y a des jours je ne sais même pas quel jour on est, ça m'arrive tout le temps, on rentre le soir à l'internat. » Le périmètre de pare-excitation que l'institution constitue par ses repères spatio-temporels demeure assez confus pour Marc. Nous pouvons penser qu'il accorde à l'Institution le soin de le guider dans un espace-temps qu'il ne mentalise pas ou peu.  

Les projets de Marc semblent être en suspens. Un projet a été mis en place auparavant, celui d'un placement dans un collège mais celui-ci n'a pas tenu : « J'étais au collège mais du coup j'y suis plus, je suis retourné à Grèzes. » Depuis, aucun projet n'a été mis en place.

\subsection{Analyse clinique}

\subsubsection{Problématique objectale et narcissique du sujet}

Selon les propos de Marc, nous pouvons supposer que celui-ci a souffert d'un manque d'apports affectifs sécurisants. Sa mère étant absente depuis ses 3 ans et son père le maltraitant, nous pouvons effectivement penser que l'édification narcissique du sujet a été atteinte. Mais Marc a été aussi l'objet de l'investissement positif d'un père marqué par l'ambivalence, à la fois protecteur mais également maltraitant. L'absence de la mère et l'ambivalence du père développent chez Marc une relation anti-objectale afin de se protéger contre l'angoisse de la perte d'objet. Nous pouvons percevoir ce mécanisme de défense anti-objectale tout au long du discours de Marc. Chaque relation à l'autre est introduite par un anti-investissement, puis par un investissement positif  notamment envers les objets qui vont lui apporter un apport bénéfique, tels son beau-père et son éducateur référent. L'ambivalence du père  semble avoir été introjecté chez Julien, comme mode relationnel. 

La construction identitaire de Julien semble être de bonne qualité comme les résultats au test de Rorschach l'indiquent. Les planches III et V, prouvent la capacité du sujet à s'identifier. Le support identificatoire majeur de Julien se réfère à son beau-père : « parce que lui c'était pareil, il ne savait pas lire ni écrire à mon âge », « « Plus tard je veux rentrer dans l'armée. Ca fait un moment que je veux rentrer. Le père de mon beau-père c'était un ancien militaire. Ils ont tous fait ça, c'est de famille. » Cependant, cet investissement identificatoire attribué à son beau-père, va toujours être contrasté par l'échec de son propre père dans sa fonction d'objet parental : « c'est pas lui qui m'aurait appris ». L'identification aux figures viriles semble avoir été acquises par Marc selon les résultats au test de Rorschach, concernant les planches IV et VI. De plus, sa volonté à intégrer l'armée plus tard, va dans le sens de la recherche d'un père, d'un cadre sécurisant, présence d'un chef visible (Freud\index{Freud}, 1921). Ce rôle qu'il effectuera auprès de sa petite sœur : « j'aime bien m'occuper d'elle ».

 Cependant, durant la petite enfance de Marc, l'absence de sa mère semble avoir marqué son fonctionnement psychique comme nous pouvons le percevoir aux résultats du test de Rorschach. Les refus se sont concentrés en direction des planches II, VII et IX. Ces planches réactivent le vécu face au maternel féminin en terme de perception et de réaction. Le sujet évoque le vide lorsqu'il est confronté à l'imago maternelle. De plus, il remet en question la fonction maternelle de sa propre mère : « t'as voulu une gamine, c'est toi qui t'en occupes ».

Malgré une bonne capacité à s'identifier, à une possibilité d'investissement de supports identificatoire, notamment concernant les figures viriles, la faiblesse de Marc se dévoile cependant par une défaillance de l'imago maternelle mais aussi par un sentiment fragile de son identité. Ceci participe du processus adolescent où le Moi se trouve déstabilisé. Ce sentiment d'identité fragilisé se dévoile pour Marc à travers ses réponses au Rorschach par une faiblesse de variété des contenus, ainsi que par un nombre élevé de réponses H et A, représentations humaines et animales, qui vont dans le sens d'un sentiment d'identité fragilisé (Chagnon\index{Chagnon} et Cohen de Lara\index{Cohen de Lara}, 2002). Le protocole de Marc étant restrictif, nous pouvons penser à une difficulté d'élaborer. Tout au long de l'entretien,  la difficulté de Marc à se situer à une place est marquée. Il paraît important pour lui de préciser à chaque indication qu'il ne se situe pas à « sa place ». Il le rappelle lorsqu'il déclare être en CE2 au lieu du CM2, être en 6ième au lieu de la 4ième, ou qu'il indique par lui-même que le jour de son passage à l'acte au sein du collège correspondait au moment de la réunion qui traitait la question de sa place. Nous pouvons interpréter cette difficulté à occuper une place comme une faille du narcissisme, ainsi qu'une brèche dans la constitution de son identité. Cette faille narcissique où son intégrité psychique semble menacée, se manifeste par le passage à l'acte.

\subsubsection{Traitement des motions pulsionnelles}

Les propos de Marc relatent son incapacité à traiter psychiquement le conflit. Lorsque celui-ci se trouve confronté à une tension provoqué par un objet, il est impossible pour lui d'élaborer ce conflit psychique. Ce défaut d'élaboration va dans le sens d'un défaut de contenant symbolique qui permettrait la pare-excitation. Ce défaut d'intégration de contenant va le pousser à agir, en cherchant une sorte de contenant afin de ne pas se retrouver confronté à un anéantissement psychique. Lorsque ses débordements pulsionnels ne sont pas dirigés vers l'équipe éducative ou les jeunes, Marc la déploie contre les murs de sa chambre. Marc est en quête de limites qui pourraient avoir fonction de contenir son agressivité libre, délimitation entre l'intérieur et l'extérieur qu'il n'a pas pu introjecter. Sa réactivité à la présentation de certaines planches au Rorschach va dans le sens de cette réactivité pulsionnelle qu'il ne parvient pas à contenir. Notamment par son comportement connoté par le refus, ponctué par des exclamations. De plus, à la planche II, qui favorise l'émergence des pulsions libidinales, Marc n'a pas su contenir son débordement pulsionnel : « j'vois rien du tout là ! Mais rien ! Pouah ! », mais il a su, après un silence, amener une réponse : « Deux personnages qui se tapent ». À la planche VI, la pulsionnalité s'exprime par la destructivité : « un chat écrasé ». Marc n'ayant pas recours à la fonction symbolique  afin de se séparer et de s'individualiser, il passera à l'acte sur la personne de l'éducateur afin de se défaire de cette position où il est un objet, tenu par le bras, dépendant physiquement de l'institution. Marc n'a pas su élaborer le conflit psychique qu'a provoqué le contact physique contenant de la part de l'éducateur sur lui. Il n'a pas été possible pour lui de se confronter à son propre système de pare-excitation car celui-ci visiblement défaillant, n'a pas pu contenir ce débordement pulsionnel.

\subsubsection{Les effets de la figuration psychique  de l'institution pour Marc}

Marc semble avoir intégré le mode d'organisation institutionnel. Chaque professionnel correspond à une activité, et sa relation avec celui-ci déterminera son rapport à l'activité concernée. De plus, selon lui, cette relation avec les professionnels va définir l'environnement et impacter son état psychique. Les éducateurs avec lesquels il semble entretenir de relations favorables, mettront en place un climat qu'il décrit comme apaisant, et donc par la-même, qui va l'apaiser lui-même. L'ambiance régie par ces éducateurs semble constituer pour Marc une fonction pare-excitatrice. Des éducateurs avec lesquels il semble maintenir des relations moins favorables vont, pour Marc, instaurer un climat agité, qui va également sembler l'agiter. Cette porosité des limites entre l'état psychique de Marc et le contexte environnemental marque une dépendance psychique assez notable mais également la reconnaissance d'une figuration psychique incarnée par l'institution.

Lorsque Marc est en situation de crise, une discussion avec son éducateur référent va venir faire pare-excitation, ce qui lui permettra, selon lui, de s'apaiser et de se calmer.

Marc semble avoir intégré les lois et les règles, ainsi que le cadre. Le cadre permettant un système de temps, de lieu et d'action par l'appareillage psychique des intervenants est un principe méthodologique d'intervention institutionnelle comme le précise Pinel\index{Pinel} (1996). Cependant, Marc n'arrive pas à se représenter ses journées dans le temps, les lieux et les actions. L'intégration  du règlement, du temps de l'horloge et de l'espace géographique participe d'une action contenante, et se répercute sur les limites du Moi, c'est ce que Bergeret\index{Bergeret} (1988) a nommé « psychothérapie institutionnelle ». Mais  cette action contenante semble être acceptée par Marc sans qu'il puisse l'intégrer car il paraît en difficulté de se repérer et d'intégrer son emploi du temps. 

Les projets de Marc sont en suspens. Depuis son échec d'intégration à un établissement scolaire Marc ne semble pas avoir été tenu par un projet. D'après Marty\index{Marty} (2007), l'institution ne va se constituer en pare-excitation que si elle est capable de contenir l'excitation pulsionnelle, de lui donner forme et limite, puis de lui donner sens. Dans le cas de Marc, aucune activité ne paraît revêtir cette caractéristique. Seule l'activité sportive est investie par Marc pour ses vertus qu'il décrit comme apaisantes et que nous pourrions qualifier de possibilité de décharge psychique. Nous pouvons mettre en lien cette absence de projet qui permettrait de contenir, maintenir et mobiliser l'excitation pulsionnelle de Marc avec l'échec de la pare-excitation de l'institution pour Marc. Bien que Marc paraisse investir l'institution en tant que figuration psychique, sa dépendance au climat institutionnel peut parfois, selon les circonstances, venir  mettre à mal  la fonction de pare-excitation de celle-ci. Lorsque Marc, en réponse à la contention physique de l'éducateur, est passé à l'acte, nous pouvons interpréter cet acte comme  une tentative de compenser une faille narcissique. Afin de se défaire de cette dépendance, qui s'est révélée être physique dans cette situation, celui-ci  a réagi par une destruction du lien à l'objet. Ne pouvant pas accéder aux fonctions symboliques permettant la séparation, celui-ci passera à l'acte. L'institution peut-être rapprochée de l'objet-mère car celle-ci permet la figuration psychique de l'adolescent, qui va en être dépendant afin d'accéder à des représentations symboliques. Cependant, étant dans le processus de l'adolescence, Marc sera à la fois dépendant de l'institution mais voudra la rejeter afin d'accéder à son individualité. Mais celui-ci n'ayant pas accès aux figurations symboliques, son seul recours afin de se représenter la séparation donc l'individuation sera le passage à l'acte, qui sera une tentative de symboliser.

\chapter{Synthèse des résultats}

\section{Synthèse des fonctionnements psychiques et de leurs rapports à l'institution en tant que figuration psychique}

Julien et Marc sont dans la période de l'adolescence. Dans le discours de Julien transparait un détachement des figures masculines antérieures afin de se forger sa propre identité. Il va s'appuyer sur des éléments de son environnement extérieurs au champ familial afin de se constituer sa propre identité. Ces éléments appartiennent au champ institutionnel, que ce soit autour de son activité professionnelle et sportive, mais également issus de son groupe de pairs. Cependant, ses propos rendent compte de l'importance accordée à sa religion qui lui permet de s'appuyer sur un sentiment d'appartenance à un autre groupe défini. Ce sentiment d'appartenance paraît fortifier son narcissisme, qui est fragilisé lors de cette période.                                                

Marc semble encore attaché aux figures parentales, notamment son beau-père qu'il investit comme support identificatoire, ce qui définit d'ailleurs son choix de projet professionnel. Le groupe de pairs dans lequel Marc se sent le mieux est celui situé à l'extérieur de l'institution, constitué également par son grand-frère. Contrairement à Julien, Marc ne semble pas investir un domaine particulier au sein de l'institution, ni son activité éducative et professionnelle, ni l'adhésion à un groupe de jeunes. Hormis exclusivement le sport, où celui-ci déclare se sentir le mieux, car cette activité possède, selon lui, des propriétés que l'on pourrait qualifier d'apaisantes. 

C'est aussi le cas de Julien qui investit le sport comme son activité de prédilection, qui lui permet de se décharger, et d'investir son corps comme une limite contenante.

C'est à l'âge de trois ans que les deux sujets ont été marqués par une faille de l'environnement, et par là même de leur relation objectale primaire,  ce qui aura un effet majeur sur leur mode de relation avec l'environnement. C'est pendant cette période que Julien a fui avec sa famille son pays d'origine en guerre. Nous pouvons dire que cet environnement insécurisant, ce déracinement brutal, début d'une quête longue et instable à la recherche d'un refuge va impacter la relation objectale primaire du sujet. Dans ce contexte anxiogène, où l'objet parental a projeté sa propre angoisse de mort sur Julien, nous pouvons dire que celui-ci a pu éprouver des difficultés à se saisir des bénéfices d'un climat fiable et sécurisant, et dans ce contexte, n'a pas pu permettre un investissement de qualité envers l'objet parental.       

La mère de Marc est partie lorsqu'il était âgé de trois ans. Il a été l'objet de l'investissement de son père, marqué par l'ambivalence, à la fois protecteur mais également maltraitant. Nous pouvons dire que Marc a souffert d'un manque d'apport affectif fiable et sécurisant. L'absence de la mère a développé chez lui une relation anti-objectale afin de se protéger contre l'angoisse de la perte d'objet, de l'abandon. L'ambivalence du père, basculant entre protection et maltraitance, n'a pas pu permettre un climat fiable et structurant. C'est en développant une relation anti-objectale que Marc peut se protéger de ce stigmate archaïque. S'il perçoit chez l'autre une possibilité d'apport affectif et sécurisant, Marc va l'interpréter comme angoissante, car pour lui, l'intention protectrice est étroitement liée à une connotation négative. 

Ces relations primaires objectales, empreintes d'angoisse, d'insécurité, d'ambivalence ou bien même totalement absentes vont impacter sur l'édification narcissique de ces sujets. Effectivement, leurs assises narcissiques n'ont pas pu être définies par une introjection correcte des objets primaires. Or, lorsque la relation primaire objectale est sécurisante, le bébé acquiert un sentiment d'intégrité qui donne à son Moi une enveloppe narcissique et un bien être de base. C'est ce que Anzieu\index{Anzieu} (1985) nomme le pare-excitant qui est une des fonctions de l'appareil psychique permettant de défendre l'effraction pulsionnelle endogène tout en contribuant à satisfaire les apports d'excitation nécessaires. Lorsque cette fonction est perturbée, l'acquisition de la délimitation entre l'intérieur et l'extérieur, entre le dehors et le dedans est perturbée. L'environnement dans la petite enfance est important  car si il y a une faille du contenant, alors le développement psychique est fragilisé et se traduit par des failles de l'établissement et de la construction du système de pare-excitation. 

Nous avons vu lors des descriptions des passages à l'acte de Marc et Julien  que leurs systèmes de pare-excitation ne sont pas opérationnels. Lorsqu'ils sont confrontés à une tension, celle-ci paraît insurmontable pour leurs appareils psychiques. Ils sont dans l'incapacité d'élaborer et de symboliser ce conflit et de trouver une solution par eux-mêmes pour se calmer ; ce qui relève d'une absence d'un système de pare-excitation. C'est pourquoi ils sont dépendants de l'environnement, car celui-ci peut se constituer en tant que pare-excitation, en limite contenante de leurs décharges. Cette agressivité libre cherche une sorte de contenant, pour échapper à l'anéantissement psychique. Nous avons vu dans notre partie théorique que l'objet-mère permet l'advenue des fonctions symboliques dans le processus de séparation et d'individuation du fait de sa capacité à se confronter à l'agressivité primaire de l'enfant. Nous pouvons dire que dans le cas de Julien et Marc, l'objet-mère n'a pas pu leur permettre l'advenue des fonctions symboliques, ce qui a inhibé pour chacun l'établissement d'un système suffisant de pare-excitation propre. Ils y recourent donc en tant que ressource externe, sont donc dépendants de l'environnement et peuvent ainsi investir l'institution comme figuration psychique. 

La caractéristique sur laquelle les fonctionnements psychiques de Julien et Marc convergent le plus est la défaillance d'élaboration psychique. Pour les deux sujets, cette décharge cherche à se heurter à une sorte de contenant. Pour Julien, elle  se dirige  vers le mobilier, fenêtre et vaisselle, et pour Marc, afin de ne pas la projeter contre les professionnels ou les jeunes, celui-ci la déploie contre le mur de sa chambre de l'institution. Dans ces moments de débordement pulsionnel, ce qui viendra faire pare-excitation pour Julien, sera la présence physique des éducateurs qui le contiendront au sol, et leurs présences verbales par leurs présentations. Pour Marc, une discussion avec un éducateur défini pourra le calmer dans ces moments-là. 

Malgré ses efforts, Julien ne se rappelle pas de moments précis où il aurait pu se mettre en colère en dehors de l'institution, il en souligne pourtant la fréquence.  Mais il se souvient immédiatement d'un exemple précis de passage à l'acte au sein de l'institution. Nous pouvons mettre en évidence que le périmètre institutionnel pour Julien se constitue en tant que figuration psychique, ce qui lui permet d'élaborer. Marc a éprouvé des difficultés à se souvenir d'épisodes où il aurait pu passer à l'acte. Il semblait difficile pour lui de différencier un passage à l'acte au sein de l'institution et un passage à l'acte en dehors de l'institution. Lorsque nous l'avons interrogé sur un passage à l'acte au sein de l'institution, celui-ci a répondu par la retranscription d'un acte violent en dehors de l'institution, notamment à l'intérieur de son collège. Les repères spatiaux de Marc paraissent confus, ainsi que ses repères temporels, lorsqu'il fait allusion à ses difficultés à se rappeler du jour et de son emploi du temps. Les repères spatio-temporels de l'institution ne permettent pas une structuration psychique chez Marc. Pour lui, la figuration psychique de l'institution se relève défaillante ce qui n'est pas le cas pour Julien. En effet, celui-ci semble avoir intégré son emploi du temps et le déroulement de ses journées. La mise en place d'un projet professionnel de la part de l'institution pour Julien, endosse le rôle d'une figuration psychique. Effectivement, celui-ci paraît être investi et réellement motivé dans son projet, ce qui va le maintenir, le contenir et mobiliser ses pulsions en lui donnant une forme et une limite pour lui donner sens, ce qui va par là-même jouer le rôle de pare-excitation. Cependant, aucun projet n'a été mis en place pour Marc, ce qui laisse la mobilisation de ses pulsions en suspens, qui ne favorisera pas la contention de ses pulsions pour leur donner forme et limite.  Ceci compromet la fonction pare-excitatrice de l'institution car celle-ci, pour lui, ne semble pas donner pas un sens à ses pulsions.

\section{Interprétation sur l'échec de la fonction de pare excitation de l'institution}

Nous pouvons dès à présent répondre à notre question initiale sur la compréhension de la mise en échec de la fonction de pare-excitation de l'institution par les sujets adolescents violents. 

Tout d'abord, nous pouvons mettre en lien l'échec de la propre fonction de pare-excitation des sujets avec leurs défaut d'élaboration, et donc de symbolisation. Effectivement, à travers leurs propos, nous pouvons nous rendre compte qu'étant dans l'incapacité d'élaborer face à un conflit psychique, ils ne pourront symboliser ce conflit. Cette défaillance symbolique est étroitement liée à un défaut de leur propre système de pare-excitation. Ils n'arrivent pas à trouver des solutions issues de leur psychisme qui leur permettrait de réguler cette tension psychique. Cette incapacité à se heurter à leur propre système de pare-excitation prouve une dépendance à l'égard de l'environnement. Cet environnement permettra de répondre et de combler cette défaillance psychique. C'est pourquoi la psychothérapie institutionnelle se propose de se constituer en tant que pare-excitatrice.

Nous pouvons mettre en lumière ce qui semble faire pare-excitation chez Julien au sein de l'institution. La capacité de l'institution à avoir pu mobiliser l'excitation pulsionnelle de Julien par son investissement en atelier mécanique constitue un élément important de pare-excitation. Son projet de stage en apprentissage lui permet de donner forme et limite à cette excitation pulsionnelle tout en lui donnant du sens. De plus, son intégration des lois, des règles et du cadre spatio-temporel de l'institution fortifie cette fonction de figuration psychique pare-excitatrice. Sa relation avec l'équipe éducative semble être sécurisante et apaisante. 

Pour Marc, son absence de projet au sein de l'institution ne lui permet pas d'être contenu  et de mobiliser ses excitations pulsionnelles. Celles-ci étant libres, elles soulignent la faille de l'institution à se constituer en tant que pare-excitatrice. Cependant, c'est à travers l'équipe éducative que Marc pourra être contenu et se confronter à la pare-excitation, par la mise en place par les professionnels d'un climat apaisant. La porosité des limites de Marc entre l'intérieur et l'extérieur, entre son psychisme et l'environnement, l'établissement et le maintien d'une ambiance apaisante aura un impact important sur sa quiétude psychique.

Cependant, un épisode d'échec de la fonction de pare-excitation de l'Institution autour d'un passage à l'acte de Julien a pu être relevé. La dépendance de ce dernier à un environnement qui peut soutenir sa structuration psychique va pendant la période de l'adolescence  l'amener conjointement  à un refus de cette même dépendance afin de pouvoir se séparer de cet objet institutionnel dans le but de s'individualiser. C'est en tant qu'effraction dans le narcissisme du sujet  que l'Institution a provoqué le passage à l'acte de Julien, donc de sa fonction pare-excitatrice. Julien s'est senti réduit en  objet de l'institution car il a vécu son identité comme attaquée dans son narcissisme mais il a accepté, et recherché, la contenance physique -codifiée par lui- que l'Institution lui a procuré.

Nous pouvons dire que le passage à l'acte de Marc a permis une destruction du lien à l'objet. Il s'est senti dépendant de l'objet et afin de sauvegarder son narcissisme fragilisé, le passage à l'acte lui a permis de récupérer son narcissisme par un moyen de se figurer la séparation de cet objet.

N'ayant pas recours aux fonctions symboliques leur permettant la séparation et l'individuation, Julien et Marc ont recouru au passage à l'acte afin de se séparer symboliquement de l'institution donc de s'individualiser, ce qui a mis en échec la fonction de pare-excitation de l'institution. Nous pouvons dire que Julien et Marc n'ont pu bénéficier de la fiabilité d'une relation objectale primaire lors de leur  enfance, ce qui leur aurait permis l'advenue des fonctions symboliques par la contenance de leur agressivité libre.  A l'adolescence,  l'institution   réactualise alors le processus de séparation et d'individuation afin de mettre en place pour chacun d'eux la fonction  symbolique. La contenance institutionnelle a un rôle de pare-excitation  permettant d'advenir aux fonctions symboliques défaillantes des sujets par un étayage de leur Moi par la figuration psychique que renvoie l'institution. 

C'est sur la toile de fond de la problématique ambivalente de l'adolescence, entre dépendance et tentative de séparation et d'individuation, que se noue la causalité du passage à l'acte. Dès le moment où Julien et Marc se sentent soumis à l'emprise de l'objet, ici institutionnel, ils veulent s'en détacher pour préserver leur narcissisme et leur individualité. Le seul moyen qui s'offre à eux est le passage à l'acte. Finalement, pour ces adolescents, le passage à l'acte revêt un espoir d'accéder à ces fonctions symboliques défaillantes pour eux et qui sont responsables de leurs narcissismes fragilisés. Julien et Marc vont adresser à l'Institution leur espoir d'accéder à ces fonctions symboliques et de retrouver ainsi un renfort narcissique. L'Institution porte ainsi le rôle de l'objet-mère. Ce rôle va être permis par l'incarnation d'une figuration psychique, qui se constitue en tant que pare-excitatrice, ce qui peut permettre à l'adolescent d'incorporer cet objet afin de se réaliser en tant qu'être psychique indépendant et individuel. Réalisation qui n'a pas pu être permise par leur relation objectale primaire.

\chapter{Discussion et conclusion}

\section{Discussion théorique et méthodologique}

L'objectif de notre recherche visait à réunir des données permettant de répondre à notre question de recherche qui consiste à comprendre la mise en échec de la fonction de pare-excitation de l'institution par l'adolescent violent. Il s'agissait alors de mettre en lumière trois aspects, tel que la clinique de l'adolescence, la clinique de l'acte mais également la clinique de l'institution.

Dans un premier temps, nous nous sommes penchés sur les travaux de plusieurs auteurs, tels que les travaux de Marcelli\index{Marcelli} et Braconnier\index{Braconnier} (1992), ainsi que ceux de Gutton\index{Gutton} (1991) et de Kestemberg\index{Kestemberg} (1981) afin de redéfinir le processus de l'adolescence. Dans le but de mettre en évidence l'importance du processus de séparation et d'individuation de l'adolescence, comme une « reviviscence archaïque » (Gutton\index{Gutton}, 1991) les auteurs tels que Malher et Blos\index{Blos} (1967), Balier\index{Balier} (1988) et Winnicott\index{Winnicott} (1956), soulignent l'importance de la relation primaire objectale pour la construction solide du Moi et de l'advenue des fonctions symboliques. Effectivement, c'est lors de cette période que va se construire le Moi. Dans notre partie théorique, plusieurs auteurs tel que Balier\index{Balier} (1988) et Bergeret\index{Bergeret} (1984) s'accordent à dire que de ce dysfonctionnement de la relation primaire objectale  va résulter  une faille narcissique,  une défaillance du Surmoi et  une désintrication pulsionnelle. Le blocage de l'élaboration archaïque et du manque de séparation entre soi et objet, va précipiter le sujet dans la violence afin de ne pas sombrer dans la dépression (Balier\index{Balier}, 1988).

Cette faille narcissique, par un blocage d'élaboration archaïque, du fait du manque de séparation entre soi et objet, va entraîner une dépendance à l'environnement qui permettra au sujet de le constituer en tant que figuration psychique.  C'est pourquoi une partie clinique de l'institution était nécessaire. Cette partie souligne le principe méthodologique d'intervention de l'institution qui se constitue en tant que pare-excitation par l'intégration d'un cadre (Pinel\index{Pinel}, 1966). D'ailleurs l'intériorisation de ce cadre constitue une partie de la pare-excitation, car nous pouvons rapprocher celle-ci de l'intégration du cadre maternel (Balier\index{Balier}, 1988). Les travaux d'Anzieu\index{Anzieu} (1985) étayent également nos recherches par son concept du « Moi-Peau » que l'on va rapprocher de l'institution par son principe de maintenance et de contenance psychique.

Afin d'analyser nos entretiens semi-directifs, nous avons élaboré une grille  comportant trois axes d'investigation. Ceci a permis pour le premier axe  le recueil de données d'éléments anamnestiques du contexte familial, du contexte scolaire et de la nature de ses activités de prédilection afin de rendre compte du fonctionnement psychique du sujet, notamment de sa relation objectale et narcissique. Dans un second temps, les données concernant son passage à l'acte et les solutions qu'il met en place afin d'apaiser ses débordements pulsionnels mettent en lumière son traitement des motions pulsionnelles. Son rapport à l'acte et son rapport à l'institution permettent d'investiguer la fonction de pare-excitation et de repérer à quel moment elle fait  échec dans la dynamique du sujet, en lien avec sa figuration psychique de l'institution. 

Notre hypothèse générale supposait d'évaluer le processus de symbolisation de séparation/individuation, le passage à l'acte violent et la mise en échec de la fonction de pare-excitation. Concernant l'opérationnalisation du passage à l'acte violent et de la mise en échec de la fonction de pare-excitation, nous pensons avoir pu saisir des données. Cependant, il a été compliqué d'opérationnaliser le processus de symbolisation de séparation et d'individuation, mais nous avons tenté de la saisir à travers l'étude de la capacité du sujet à faire appel à ses propres ressources psychiques ou à une figuration psychique. Ces données ont pu alors vérifier si le sujet était capable d'élaborer seul la question de la séparation et de l'individuation, ou bien si son fonctionnement psychique dépendait de l'environnement, ce qui suppose une défaillance au niveau du processus de séparation et d'individuation.

 Nous pouvons dire que l'élaboration de cette grille d'entretien a pu nous permettre d'obtenir des données de réponse concernant notre question de recherche. Bien que nous pensons qu'une grille d'entretien plus étoffée aurait pu faciliter et rendre plus fiable notre analyse. De plus, nous avons tenu à effectuer une passation au test de Rorschach que nous avons jugé utile dans le recueil des données concernant le fonctionnement psychique. Effectivement, les protocoles du Rorschach de Marc et Julien, ont provoqué un questionnement quant à la validité de ces protocoles, dû à la pauvreté et à la restriction des contenus. Cependant, par leurs contenus plutôt faibles et leurs refus significatifs, ont pu prouver une difficulté d'élaboration, une porosité des limites entre le monde interne et l'environnement, une dépendance de l'environnement, une défaillance de la pare-excitation et enfin une incapacité à traiter psychiquement le conflit pulsionnel. Ces données ont pu confirmer et conforter ce que nous avions déjà repéré lors de notre analyse de l'entretien semi-directif. De plus, les refus ont été notés à des planches significatives, qui ont pu donner du sens.

\section{Discussion concernant la position du chercheur sur le terrain}

Notre intervention au sein de l'ITEP (Institution Thérapeutique Éducatif et Pédagogique) s'est globalement bien déroulée. L'accord de l'équipe de direction de l'institution et de la psychologue s'est effectué dans les temps voulus. Sur le terrain, la psychologue nous a bien orienté. La première rencontre avec les jeunes afin de présenter notre projet de recherche s'est passée de façon positive et a même suscité de la curiosité de leur part. Après leur accord, nous nous sommes tenus aux règles institutionnelles et avons prévenu l'éducateur référent, l'éducateur du groupe du jeune ainsi que les professeurs de l'atelier et de la classe. Nous avons pris le soin également de demander l'autorisation de leur psychologue attitrée. La rencontre avec le premier jeune, Julien, s'est très bien déroulée, il est arrivé par lui-même à la date, à l'heure et au lieu du rendez-vous comme convenu. Après l'entretien, nous sommes allés à la rencontre de son éducateur référent afin de lui faire un retour positif. Son éducateur référent nous a expliqué avant l'entretien que Julien était dans une phase « calme » et nous a fait partager sa crainte que notre entretien suscite chez le jeune, un débordement pulsionnel qui aurait pu se répercuter sur le cadre institutionnel. C'est pourquoi nous lui avons proposé de lui faire partager un retour concis, sans déroger à la règle de la confidentialité, afin de ne pas provoquer de rupture de liens institutionnels. 

Notre rencontre avec Marc  a été plus complexe. Bien que celle-ci se soit passée correctement lors de la première entrevue, après son accord favorable, Marc ne s'est pas présenté pour la seconde. Nous sommes passés par son éducateur référent qui nous a indiqué que celui-ci était en activité sportive. Après cette activité, je suis allée à la rencontre de Marc, qui nous a affirmé avoir oublié et ne plus vouloir poursuivre notre recherche (c'était la fin de journée). C'est pourquoi nous lui avons proposé un rendez-vous le lendemain matin, tout en précisant qu'il n'était pas obligé de s'y rendre. Finalement, Marc s'est présenté le lendemain matin pour la deuxième entrevue.

Nous avons fait le choix de ne pas enregistrer les entretiens. Pour ces adolescents éprouvant un sentiment d'insécurité et craignant une menace d'intrusion, le choix de l'enregistrement aurait pu, à notre sens, biaiser la qualité des résultats en gênant  l'instauration d'une nécessaire relation de confiance.  Cependant, il est important de souligner que la retranscription des entretiens a tendu à être le plus possible objective grâce à une prise de notes saisissant les mots choisis par le sujet afin de ne pas transformer son discours. Notre impossibilité à retranscrire les propos de manière fidèle, pourrait soulever des débats quant à l'éthique et à la rigueur de notre recherche.

Enfin, il nous a été difficile de soutenir la place de chercheur auprès de ces jeunes en souffrance et de garder la bonne distance afin de préserver l'objectivité et le déroulement de notre recherche. L'entretien avec Marc a été difficile à mener, et nous avons eu des difficultés à préserver le bon déroulement de notre entretien de recherches par ses digressions. Sa volonté de vouloir partager sa souffrance engendrée par l'absence de son père, nous a situé dans une position très inconfortable où il était difficile de ne pas intervenir.

\section{Discussion empirique}

L'interprétation des résultats nous a apporté plusieurs éléments de compréhension sur ce qui faisait pare-excitation chez les adolescents de la part de l'institution, mais également sur ce qui faisait échec de cette fonction. Nous avons tenté de saisir à quel moment psychique du sujet correspondait cet échec et le rôle que pouvait tenir sa propre figuration psychique de l'Institution.

Nous nous sommes rendu  compte que la défaillance d'élaboration de l'adolescent  ne pouvait pas lui permettre l'accès à la capacité de symbolisation et  renvoyait donc à une incapacité de se figurer son propre moyen de pare-excitation. C'est pourquoi nous avons pu relever chez ses sujets des similitudes au niveau de la dépendance à l'environnement qui traduisait une faille narcissique. Ces problématiques narcissiques convergent dans un premier temps, par le phénomène de l'adolescence, mais s'expliquent aussi par leur prime-enfance. Nous pouvons relever chez ces deux sujets, un dysfonctionnement des premiers liens objectaux qui s'est traduit par un environnement primaire défaillant. L'un a grandi dans un contexte environnemental anxiogène où l'objet primaire lui a projeté son angoisse de mort, ce qui n'a pas permis une fiabilité de celui-ci. L'autre, a grandi dans une famille où l'absence de la mère et l'ambivalence du père, n'ont également pas pu permettre un climat propice à la sécurité psychique. La défaillance narcissique trouve son origine dans la défaillance des premiers objets d'amour, et donc dans l'impossibilité du sujet à introjecter les qualités de l'objet.

Concernant leur passage à l'acte, nous pouvons dire dans les deux cas qu'il consiste  à   défendre leur narcissisme fragilisé. Pour l'un, il tentera de sauvegarder son identité en réagissant par le passage à l'acte à une effraction de l'objet dans son narcissisme. Pour l'autre, il va détruire le lien à l'objet, afin de nier sa dépendance tout en préservant son narcissisme. 

Dans ces deux situations, nous avons pu interpréter le passage à l'acte comme un moyen pour le sujet de tenter de symboliser la séparation et l'individuation. Notre analyse mettant en lumière ce qui fait pare-excitation pour les jeunes au sein de l'institution permet de se rendre compte de la perméabilité des limites des sujets entre le dedans et le dehors. 

L'intériorisation du cadre institutionnel par ces deux sujets, comparable à celui du cadre maternel (Balier\index{Balier}, 1988), que ce soit par l'intégration des lois et des règles, ou bien celle des repères spatio-temporels pour l'un et des repères des fonctions de chaque professionnel pour l'autre tente de se constituer en pare-excitation.   

L'institution, en tant que réactualisation de l'objet primaire, permettra à l'adolescent dans le passage à l'acte violent, de se figurer une contenance psychique pare-excitatrice, tel que l'objet-mère. Mais lorsqu'il se sentira trop dépendant de cet objet mère, l'adolescent violent n'ayant pas recours à la fonction symbolique passera à l'acte afin de tenter la symbolisation de la séparation et de l'individuation.

Nous espérons que ce mémoire aura pu fournir quelques éléments de réponses concernant l'échec de la fonction de pare-excitation, même si il paraît évident que deux sujets ne permettent pas de répondre objectivement à notre question de recherche. C'est pourquoi, si nous devions approfondir notre étude, nous élargirions notre échantillon de sujets, mais également nous construirions une grille d'entretien beaucoup plus détaillée qui permettrait un recueil de données beaucoup plus complètes.

Pour clore ce mémoire, nous espérons que celui-ci aura éclairé notre question de recherche concernant la compréhension de la mise en échec de la fonction de pare-excitation de l'institution par les sujets adolescents inscrit dans le passage à l'acte. Nous avons été surpris de l'intérêt porté à notre recherche par ses jeunes, et à leur volonté d'éclairer leur passage à l'acte. Cela nous a dévoilé clairement une souffrance palpable, qui peut-être cacherait une demande thérapeutique non-formulée.

Bien que nous n'avons pas été à l'aise avec la position de chercheur, ce mémoire nous a conforté dans notre volonté de devenir psychologue clinicien.
\renewcommand{\indexname}{Index onomastique}
\printindex
\end{spacing}
\end{document}
