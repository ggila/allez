\documentclass[a4paper, 12pt]{article}
\begin{document}

\title{Memoire}
\author{Hermine Blanchard}
\date{\today}
\maketitle

\section{Introduction}
\label{intro}
L’institution, par ses règles et son cadre institutionnel incarnant une instance symbolique contenante, peut parvenir à aider les personnes qui constituent son  public à contenir leurs débordements pulsionnels. Cependant, certaines d’entre elles semblent n’y pas parvenir à certains moments.

\section{Partie theorique}

\subsection{Clinique de l’adolescent}

\subsubsection{Les remaniements psychiques de l’adolescence : les principaux aspects dynamiques}
L’excitation sexuelle.
La problématique du corps.
Le Narcissisme.
La place de l'Idéal du Moi à l'adolescence
Identité et Identification.

\subsubsection{Les remaniements psychiques de l’adolescence : La question de la séparation et de l’individuation}
De la petite enfance à l’adolescence.
De l’importance du premier lien objectal primaire à l’advenue des fonctions symboliques.
Les effets pathologiques d’une défaillance du premier lien objectale primaire.
La dépendance à l’environnement.
L’aménagement défensif de l’adolescent en souffrance : La mise en acte.

\subsection{Clinique de l’acte déviant : la violence}

\section{Partie Methodologique}
refer to \ref{intro} on \pageref{intro}
\end{document}
